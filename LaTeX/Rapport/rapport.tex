\documentclass[11pt]{article}
\usepackage{amsmath}
\usepackage{amsthm}
\usepackage{graphicx}
\usepackage{hyperref}
\usepackage[utf8]{inputenc}
\usepackage{enumitem}
\usepackage{cancel}
\nocite{*}

\newcommand{\A}{{\mathbb A}}
\newcommand{\Evp}{{Ev(\bar{p})}}
\newcommand{\CTLf}{{CTL$^f$ }}
\usepackage[a4paper, total={6in, 9in}]{geometry}

\bibliographystyle{plain}

\title{MPRI Internship Report}
\author{Anatole Leterrier, supervised by Sam van Gool, IRIF}
\date{2022–07-2022}

\begin{document}
\maketitle

\section*{}

\subsection*{General context}
Temporal logics are at the heart of the domain of verification.
When reasoning about the behavior of some program, not only do we want 
to know which properties are true in the present, but we also need 
to be able to predict what will happen at some time in the future --- 
or which facts will remain forever. As  with any logic, to use them  
in actual verification systems, one first needs to prove their completeness 
with regards to those systems, that is, that 
their algebraic axiomatization actually models the true behavior 
of the structure. Such proofs often involve the notions of Stone duality, 
of Büchi automata and of tableaux.

\subsection*{Problem studied}
My supervisor Sam van Gool proved the completeness of a fragment of LTL
(Linear Temporal Logic) with regards to specific semantics of Kripke
structures, along with Silvio Ghilardi. Later, they presented a tableaux-based
proof of the completeness of \CTLf(a Computational Tree Logic with
fairness constraints). In both proofs, Stone duality makes for an 
impressive tool in order to reason over axioms as properties of
relations over Kripke structures. 

However, a setback of Stone ultrafilters
is that generally, those are not sets of formulas which can be characterized
by a finite subset. This makes model computation non-constructive,
and does not allow for an algorithmic solution to the satisfaction problem. 
Therefore, finding a way of making those proofs constructive may represent
an improvement, and might provide further information about how these
logics act.

\subsection*{Proposed contributions}
A large part of this internship has revolved around LTL and the several
ways of handeling completeness. It all comes down to the Satisfiability 
(SAT) problem. This has multiple approaches using Büchi automata which
recognize particular LTL formulas, 
so one thing I did was help clarify the link between states of the 
automata and atoms in a particular algebra generated by the formula.
This allows to stay closer to the essential notion of Stone duality.
This is also why I have re-explained Ghilardi and van Gool's LTL proof,
in terms of Stone duality extended to `LTL-algebras'. Other approches 
to SAT make use of tableaux, structures which are both algorithmic-friendly,
and very close to the research of models for formulas. **** 's tableaux
algorithm for solving SAT in LTL is a most recent work in this vein,
which I implemented in OCaml. 

On the other hand, Ghilardi and van Gool's proof for completeness of \CTLf
involves tableaux, but for it to be implemented in a SAT solver, one
would need to make the proof constructive. To take steps in that
direction, I have studied `\CTLf algebras' in order to characterize
\CTLf formulas by atoms.


\subsection*{Arguments supporting their validity}
All the work provided during this internship has shown that bringing down
formulas in temporal logics to a set of atoms in an algebra is a fruitful
way to prove staple results, and to understand deeply what a formula
implies in order for a model to satisfy it. Beside that, (working LTL solver
blabla). The approach to the two logics I've studied tends towards a general
understanding of the algebras in question, which lays the foundations for
adapting the best methods to more complex temporal logics.



\subsection*{Summary and future work}
CTL w/ counters ?

\end{document}