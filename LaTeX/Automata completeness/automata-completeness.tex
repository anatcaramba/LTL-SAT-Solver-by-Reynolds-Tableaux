\documentclass{article}
\usepackage{amsmath,amssymb,amsthm}
\usepackage{a4wide}
\usepackage[utf8]{inputenc}

\usepackage{setspace}
\onehalfspacing

\newcommand{\onto}{\twoheadrightarrow}
\newcommand{\inv}{^{-1}}


\newcommand{\X}{\mathbf{X}}
\newcommand{\F}{\mathbf{F}}
\newcommand{\G}{\mathbf{G}}
\newcommand{\I}{\mathbf{I}}
\newcommand{\cF}{\mathcal{F}}
\renewcommand{\S}{\mathcal{S}}
\renewcommand{\phi}{\varphi}
\title{Completeness via automata}
\author{S.~v.~Gool and A.~Leterrier (and S.~Ghilardi?)}
\newtheorem{theorem}{Theorem}
\newtheorem{lemma}[theorem]{Lemma}
\begin{document}

\maketitle
The aim of this note is to show that generalized Büchi automata can be used for
proving completeness of an axiomatization of (a fragment of) LTL.

We consider formulas (i.e. terms) in the algebraic signature with constant
symbols $\I$ and $\bot$, unary symbols $\X$, $\F$ and $\neg$ and a binary symbol
$\wedge$, and a fixed finite number of variables $p_1, \dots, p_k$. 

The axioms of our theory are the Boolean algebra axioms, together with axioms
that say $\X$ is an endomorphism, $\X\I = \bot$, $a \neq \bot$ implies $\I \leq
\F a$, and $\F p$ is the least fixpoint of the function $x \mapsto p \vee \X x$.
As usual, $a \vee b$ abbreviates $\neg (\neg a \wedge \neg b)$ and $\G a$
abbreviates $\neg \F \neg a$. We call a model of this theory an \emph{algebra}.
As always with a theory, when we have an algebra $A$ and an interpretation of
the variables $p_1, \dots, p_k$ as elements $a_1, \dots, a_k$ of the algebra
$A$, any formula $\phi$ gets a denotation in $A$, that we denote by $\phi^A(a_1,
\dots, a_k)$.  A formula $\phi$ is called \emph{consistent} if there exists an
algebra $A$ and elements $a_1, \dots, a_k$ of $A$ such that $\phi^A(a_1, \dots,
a_k) \neq \bot$ in $A$.

Note that $\mathcal{P}(\omega)$ is an example of such an algebra, with $\I =
\{0\}$, $\X$ the predecessor function on sets, and $\F$ the down-set function.
We denote by $\Sigma$ the finite set $\{0,1\}^k$. A function $w \colon \omega
\to \Sigma$ is called an \emph{infinite $\Sigma$-word}. An infinite
$\Sigma$-word defines a \emph{semantics function} $\bar{w}$, because for any $i
\in \{1, \dots, k\}$ we have a subset $\bar{w}(p_i) := \{t \in \omega \ \colon \
w(t)(i) = 1\}$, and thus any formula $\phi$ gets a value $\bar{w}(\phi)$ in the
algebra $\mathcal{P}(\omega)$, by induction on the term  $\phi$. We use the
common notation $w, t \models \phi$ to mean $t \in \bar{w}(\phi)$. We also write
$w \models \phi$ to mean $w, 0 \models \phi$. We call a formula $\phi$
\emph{satisfiable} if there exists an infinite $\Sigma$-word $w$ such that $w
\models \phi$. Note that satisfiable formulas are consistent, because
$\mathcal{P}(\omega)$ is an algebra.

We aim to prove the completeness theorem, which says that consistent formulas
are satisfiable. Let $\phi$ be a consistent formula and $A$ an algebra
witnessing this fact. 


Write $\S(\phi)$ for the finite set of subformulas of $\phi$, and $\S^+(\phi)$
for the smallest set of formulas containing $\S(\phi)$, each of the variables
$p_1, \dots, p_k$, and having the property that, for any $\F\psi$ in
$\S^+(\phi)$, also $\X\F\psi \in \S^+(\phi)$.  Let $B$ the smallest Boolean
subalgebra of $A$ that contains $\psi^A$ for every $\psi \in \S^+(\phi)$. Notice
in particular that if $\X \psi \in \S^+(\phi)$ then $\X \neg \psi^A = \neg \X
\psi^A$ is in $B$. We will henceforth just write $\S^+$ for $\S^+(\phi)$, since
$\phi$ is fixed.

Note that $B$ is a finite Boolean algebra, so it is isomorphic to
$\mathcal{P}(Q)$ where $Q$ is the set of atoms of $B$. We therefore use the
following notational convenience. When $\psi$ is a formula in $\S^+$ and $q$ is
an atom of $B$, we write ``$q \in \psi$'' as shorthand for ``$q \leq \psi^A$''.
We define a binary relation $R$ on $Q$ by

\[ q {R} q' \iff \text{ for all } \X \psi \in \S^+, \text{ if } q \in \X \psi
\text{ then } q' \in \psi.\]

Note that, in fact, if $q {R} q'$, then the following properties also hold:
\begin{enumerate} 
\item If $\X \psi \in \S^+$, then if $q' \in \psi$ then $q \in
	\X \psi$.  

\item If $\F \psi \in \S^+$, then $q \in \F \psi$ iff $q \in \psi$ or $q' \in \F
\psi$.  
\end{enumerate} 

The proof is an exercise in using the axioms that hold
in the algebra $A$. For example, if $q \not\in \X \psi$, then by definition $q
\nleq \X \psi^A$, so $q \leq \neg \X \psi^A$ since $q$ is an atom. Since $\X$ is
an endomorphism we have $q \leq \X \neg \psi^A$, and thus $q' \leq \neg \psi^A$
because $(q,q') \in R$.  The second item is left to the reader and uses the fact
that $\F \psi = \psi \vee \X \F \psi$ in $A$.

\begin{proof}
	By construction, since $\F \psi \in \S^+$, $\X\F\psi\in\S^+$.
	Assume that $q\in\F\psi = \psi\vee\X\F\psi$.
	By definition, $q \leq\F\psi^A$. As a consequence, either $q\leq\psi^A$, or $q\leq\X\F\psi^A$
	(as an atom, $q$ cannot be equal to $\F\psi^A$).
	This implies that $q'\leq\F\psi^A$, which proves our point in one way.
	For the other way, either $q\leq\psi^A$, in which case $q\leq\F\psi^A$ quite obviously,
	or $q'\leq\F\psi^A$, which by point 1. actually means that $q\leq\X\F\psi^A$. This allows to conclude.

\end{proof}

We also define a function $\sigma \colon Q \to \Sigma$ by sending $q \in Q$ to
the vector $\sigma(q) \in \Sigma = \{0,1\}^k$ defined by $\sigma(q)(i) = 1$ iff
$q \leq a_i$.

A transition relation $\delta \subseteq Q \times \Sigma \times Q$ is now
defined by $(q, s, q') \in \delta$ iff $(q, q') \in R$ and $\sigma(q) = s$. 

Let $w$ an infinite $\Sigma$-word. A \emph{run} on $w$ is an infinite $Q$-word
$\rho$ with the property that $(\rho(n), w(n), \rho(n+1)) \in \delta$ for every
$n \in \omega$. Because of the special definition of $\delta$, note that any run
$\rho$ can only be a run on the word $w := \sigma \circ \rho$. Thus, whenever
$\rho$ is an infinite $R$-path, it is a run on $\sigma \rho$, and on no other
word.
Also, we call $\rho(0)$
the \emph{starting point} of $\rho$ and if $\rho(0) = q_0$ then we say $\rho$ is
a run \emph{from} $q_0$.

Recall that the acceptance requirement of a generalized Büchi automaton is given
by a set $\cF \subseteq \mathcal{P}(Q)$ of final regions, and  that a run $\rho
\colon \omega \to Q$ is successful if, for every $F \in \cF$, $\rho^{-1}(F)$ is
infinite. In our case, we define for every formula $\F \psi$ in $\S^+$ a final
region $\neg \F \psi \vee \psi$. More formally:
%
\[ \cF := \{ \{q \in Q \ \colon \ q \leq (\neg \F \psi \vee \psi)^A \} \ \colon
\ \F \psi \in \S^+ \}.\]
%
We thus obtain a tuple $(Q, \Sigma, \delta, \cF)$ that will form the blueprint
for our Büchi automata, in the sense that they are always based on this tuple,
but have varying initial states.

\begin{lemma}\label{lem:automaton-semantics}
For every $\psi \in \S^+$ and every successful run $\rho$, we have 
\[ \sigma \rho \models \psi \iff \rho(0) \in \psi.\] 
\end{lemma}
%
\begin{proof}
By induction on the complexity of $\psi$, proving the slightly stronger
statement that for any $t \in \omega$, 
%
\[ \sigma \rho, t \models \psi \iff \rho(t) \in \psi. \]
%
When $\psi$ is a variable $p_i$, the left hand side says $t \in
\overline{\sigma\rho}(p_i)$ which means by definition that $\sigma\rho(t)(i) =
1$ and by definition of $\sigma$ this means that $\rho(t) \leq a_i = (p_i)^A$.

The cases $\I$, $\wedge$ and $\neg$ are easy. Also, when $\X \psi \in
\mathcal{S}^+(\phi)$, we have $\sigma\rho, t \models \X \psi$ iff $\sigma\rho, t
+ 1 \models \psi$, which is equivalent by the induction hypothesis to $\rho(t+1)
\in \psi$. The definition of $R$ and the first remark following it show that
this is equivalent to $\rho(t) \in \X \psi$, because $(\rho(t), \rho(t+1)) \in
R$.

Finally, for the $\F$-case, suppose $\F \psi \in \S^+$. If $\sigma \rho, t
\models \F \psi$, pick $t' \geq t$ such that $\sigma \rho, t' \models \psi$. By
induction, this is equivalent to $\rho(t') \in \psi$. We now show with a
sub-induction on  the non-negative number $\ell := t' - t$ that $\rho(t) \in \F
\psi$. When $\ell = 0$, $\rho(t) \in \psi$, and $\psi \leq \F\psi$ in $A$, so
$\rho(t) \in \F\psi$. For the induction step, we have that $\rho(t + 1) \in \F
\psi$ by the induction hypothesis. Then $\rho(t) \in \X\F\psi$ because
$(\rho(t), \rho(t+1)) \in R$, so we get $\rho(t) \in \F\psi$ since $\X \F \psi
\leq \F \psi$.

Conversely, suppose that $\rho(t) \in \F\psi$. We show that there exists $u \geq
t$ such that $\rho(u) \in \psi$, which will suffice, because then the induction
hypothesis gives $\sigma\rho, u \models \psi$, so $\sigma\rho, t \models \F
\psi$ by definition.  Since $\rho$ is a successful run and $\F \psi \in \S^+$,
pick $t' \geq t$ such that $\rho(t') \in \neg \F \psi \vee \psi$.  If $\rho(t')
\in \psi$, then we just pick $u := t'$. Otherwise, we have that $\rho(t') \in
\neg \F \psi$. In particular, $t \neq t'$, and choose $t \leq u < t'$ to be the
maximal point such that $\rho(u) \in \F \psi$. Then $\rho(u+1) \in \neg \F
\psi$, and therefore $\rho(u) \in \X \neg \F \psi = \neg \X \F \psi$. Since $\F
\psi \wedge \neg \X \F \psi \leq \psi$ (by a simple Boolean algebra manipulation
of the equality $\F \psi = \psi \vee \X \F \psi$), we now get that $\rho(u) \in
\psi$.  \end{proof}

We now use Lemma~\ref{lem:automaton-semantics} to prove the following.
\begin{theorem} For any word $w \in \Sigma^\omega$ and $q_0 \in Q$, the
	following are equivalent: \begin{enumerate} \item The automaton $(Q,
		\Sigma, \delta, \cF)$ has a successful run on $w$ starting from
	$q_0$; \item For every $\psi \in \S^+$ such that $q_0 \in \psi$, $w
\models \psi$.  \end{enumerate} \end{theorem} \begin{proof}

(1) $\Rightarrow$ (2). If the automaton has such a successful run $\rho$
starting from a state $q_0$, then by the remarks above, we have $w =
\sigma\rho$, and the right-to-left direction of
Lemma~\ref{lem:automaton-semantics} gives that if $q_0 \in \psi$ then
$\sigma\rho \models \psi$. 

(2) $\Rightarrow$ (1). Conversely, suppose that $w \models \psi$ for every $\psi
\in \S^+$ such that $q_0 \in \psi$.  Define a function $\rho_w \colon \omega \to
Q$ by: 
%
\[ \rho_w(t) := \bigwedge \{ \psi^B \ \colon \ \psi \in \S^+, w, t
		\models \psi\} \wedge \bigwedge \{ \neg \psi^B \ \colon \ \psi
	\in \S^+, w, t \not\models \psi\}.\] 
%
Notice that each $\rho_w(t)$ is indeed an atom of $B$, as it is exactly
the meet of the intersection of an ultrafilter with the finite Boolean
algebra $B$. Also $\rho_w(0) = q_0$ follows from the assumption that $w,
0 \models \phi$ for all $\psi \in \S^+$ with $q_0 \in \psi$.

The function $\rho_w$ defines a run on $w$: if $\rho_w(t) \in \X \psi$ for some
$\X \psi \in \S^+$, then this means that $w, t \models \X \psi$, so $w, t + 1
\models \psi$, and hence $\rho_w(t+1) \in \psi$.  Also, $\sigma\rho_w(t) =
w(t)$, because the variables occur in $\mathcal{S}^+$.

Finally, the run $\rho_w$ is successful. Indeed, suppose that $\F \psi \in
\S^+$. We need to show that $\rho_w$ is in the final region $\neg \F \psi \vee
\psi$ infinitely often. Consider the set \[ T := \{t \in \omega \ \colon \ w, t
\models \psi\}.\] We distinguish two cases, according to whether $T$ is finite
or infinite. If $T$ is finite, then pick $t_0$ such that  $w, t \not\models
\psi$ for all $t \geq t_0$. But then $\rho_w(t_0) \in \neg \F \psi$, so
$\rho_w(t) \in \neg \F \psi$ for every $t \geq t_0$. If, on the other hand, $T$
is infinite, then $\rho_w(t) \in \psi$ infinitely often, and we are also done.
\end{proof}

The above theorem almost proves completeness but not quite. One step is missing;
the following question needs to be answered:

{\bf Question.} Let $q \in Q$. Does there always exist a successful run starting
from $q$?

If the answer to this question is positive, then the following argument
completes the proof. Recall that the algebra $A$ witnesses the consistency of
$\phi$, and $\phi^A$ is a non-bottom element of $B$, so we can pick an atom $q
\in Q$ such that $q \leq \phi^A$, i.e., $q \in \phi$. By the positive answer to
the question, pick a successful run $\rho$ starting from $q$. Then by the
direction (1) $\Rightarrow$ (2) of the above theorem, we have $w \models \phi$,
where $w$ is the word $\sigma \rho$.

\subsection*{Existence of a successful run}
 There are several ways we could attend to prove the existence of a successful run. A first idea is to reason inductively on the number of
 elements in the terminating condition $\cF$.

 \begin{lemma}
	 For every $\phi$ such that $\cF$ is empty, there exists a run starting from every atom $q_0 \in B_\phi$. 
 \end{lemma}
 \begin{proof}
	 Since there are no eventuality conditions, the definition of termination only asks that we get an infinite path starting from $q_0$.
	 In other words, we require that every state $q_0$ has a successor in $A_\phi$. For this, consider the set:
	 \[D= \uparrow\{\psi\mid q_0\in\X\psi, \X\psi\in B\}\]
	 D is an proper filter of $B$: it is a non-empty ($\X\top=\top$ so that $\top\in D$) upset not containing $\bot$, as $q_0$ is an atom
	 and $\X\bot=\bot$;
	 and if $a_1, a_2$ are elements of $D$, we have $\psi_1\leq a_1, \psi_2\leq a_2$ such that both $\psi_i$ are in $\uparrow q_0$.
	 The latter being a filter implies that $\psi_1\wedge\psi_2 \in D$ ($\X(\psi_1\wedge\psi_2)$ is in $B$ by virtue of $\X$ being 
	 an endomorphism.). A fortiori $a_1\wedge a_2 \in D$.

	Therefore there is an atom $q_1$ below $D$. And now $(q_0,q_1)\in R$, quite obviously from definition. This is what we wanted to prove.
 \end{proof}

\end{document}
