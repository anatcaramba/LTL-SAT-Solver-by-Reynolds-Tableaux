\documentclass[11pt]{article}
\usepackage{amsmath}
\usepackage{amsthm}
\usepackage{graphicx}
\usepackage{hyperref}
\usepackage[utf8]{inputenc}
\usepackage{enumitem}
\usepackage{cancel}
\usepackage{latexsym}
\usepackage{amsfonts}
\usepackage{bussproofs}
\usepackage{tikz}
\nocite{*}

\newcommand{\A}{{\mathbb A}}
\newcommand{\Evp}{{Ev(\bar{p})}}

\bibliographystyle{plain}
\title{Monadic Second Order Logic as the Model Companion of Temporal Logic; a clearer version}
\author{Anatole Leterrier}

\newtheorem{definition}{Definition}[section]
\newtheorem{theorem}[definition]{Theorem}
\newtheorem{proposition}[definition]{Proposition}
\newtheorem{lemma}[definition]{Lemma}

\begin{document}
\maketitle

This is nothing more than another version of the 2016 paper by Ghilardi and van Gool. Since the proofs in \cite{GhivG16} are mostly developed in the appendix, rewriting them completely in a more readable way is a quite useful exercise. Doing so, my goal is to understand deeply how the constructions work, in order to produce my own version for finite trees with counters by the end of my internship. If I explicit some point more than it was in the original paper, or if I change the order of a few definitions, it is because my understanding benefits from it. Conversely, if I move quickly on a topic, it is because it is already clear in my head.
\section{Syntax and Semantics of fair CTL}\label{sec:synt_sem}

Fair CTL, which is an extension of the well-known computational tree logic, is defined intuitively so it can simulate the termination condition for executions of tree automata. For example, the definition of $EG$ is close to what we expect for a fair execution, ie. a branch on which a formula is true infinitely often (or a final state visited infinitely often). 

We will define a \emph{syntax} for the logic $CTL^{f}$, which will allow to interpret formulas as elements of a particular type of Boolean algebra, and to use results on duality for those algebras. 

On the other hand, we will define \emph{semantics}, which will allow to interpret such formulas on models, ie. trees whose nodes are labeled by sets of propositional variables supposed to hold on each node. Next section will then be dedicated to prove that these semantics are \emph{complete} with respect to the syntax, in other words, that for every formula $\varphi$ that holds in every model, we can syntactically deduce that $\varphi$ is equivalent to the $\top$ formula of the logic.

\subsection*{Syntax of fair CTL}\label{subsec:syntax_CTLf}

\begin{definition}\label{CTLf_formulas}
    Let $\bar{p}= \{p_1,...,p_n \}$ be a finite set of propositional variables. We define inductively a \emph{$CTL^{f}$-formula} $\varphi$ to be of the shape:
    \begin{itemize}
        \setlength\itemsep{0em}
        \item[-] $\bot$
        \item[-] $p \in \bar{p}$
        \item[-] $\neg \varphi$
        \item[-] $\varphi \vee \psi$
        \item[-] $\diamond \varphi$
        \item[-] $EU(\varphi,\psi)$
        \item[-] $EG(\varphi,\psi)$ 
    \end{itemize}
    For convenience, we then also define the De Morgan duals of our binary operations, respectively $\wedge$ for $\vee$; $AR$ for $EU$; $AF$ for $EG$. For the unary operator, we set $\Box \varphi = \neg \diamond(\neg \varphi)$; Finally, we set $\top$ to be the negation of $\bot$, naturally.
\end{definition}


\begin{definition}\label{quasi_eq_CTLf}
    We define the \emph{quasi-equational theory $CTL^{f}$} by requiring the formulas to satisfy the following axioms:
    \begin{enumerate}
        \setlength\itemsep{0em}
        \item Boolean algebra axioms for $\vee,\neg,\bot$;
        \item Unary operator $\diamond$ preserves finite joins, including the empty joint $\bot$, making $CTL^f$ into a modal logic;
        \item $\diamond \top = \top$
        \item Binary operators $EU$ and $EG$ satisfy the following \emph{fixpoint axioms} for all $a,b,c$:
        \begin{itemize}
            \item[-] $a \vee (b \wedge \diamond EU(a,b)) \leq EU(a,b)$
            \item[-] $a \vee (b \wedge \diamond c) \leq c \implies EU(a,b) \leq c$ 
            \item[-] $EG(a,b)\leq a\wedge \diamond EU(b\wedge EG(a,b),a)$
            \item[-] $c\leq a\wedge \diamond EU(b\wedge c,a) \implies c \leq EG(a,b)$
        \end{itemize}
        In other words, $EU(a,b)$ is the \emph{least pre-fixpoint} of the function $x \mapsto a \vee (b \wedge \diamond x)$.
        And $EG(a,b)$ is the \emph{greatest post-fixpoint} of the function $x \mapsto a\wedge \diamond EU(b\wedge x,a)$.
    \end{enumerate}
\end{definition}

Importantly, we remark that for all $a,b$, $AR(a,b)$ is the greatest post-fixpoint of $x \mapsto a \wedge (b \vee \Box x)$, while $AF(a,b)$ is the least pre-fixpoint of $x \mapsto a \vee \Box AR(b\vee c,a)$.

What is called a \emph{quasi-equational theory} corresponds to what we expect of a syntax. That is, there are \emph{equations} (equalities) which correspond to axioms of a deduction system, like: \AxiomC{}\UnaryInfC{$\top\vdash\diamond\top$} \[\DisplayProof\]
But there also are \emph{quasi-equations} (implications) which translate to deduction rules: \AxiomC{$c\vdash a\wedge \diamond EU(b\wedge c,a)$}\UnaryInfC{$c \vdash EG(a,b)$}\[\DisplayProof\]
However, we rarely (or never) use the rules properly speaking, but we define a structure which embodies those properties and use operations on it. This is a step in advance towards semantics, in some sense.
\begin{definition}\label{CTLf-algebra}
    Given what is said above, we can deduce the definition of a \emph{$CTL^f$-algebra}, which is a tuple $\A=(A,\bot,\neg,\vee,\diamond,EU,EG)$ which verifies every axiom of $CTL^f$ quasi-equational theory.
\end{definition}

Also, the way EU and EG formulas are defined reminds the fixpoint operators of the \emph{$\mu$-calculus}. For example, for every $a,b$, $EU(a,b)$ could have been defined as $\mu x.(a \vee (b \wedge \diamond x))$, the least fixpoint of this monotone function (it is enough to say the least \emph{pre-fixpoint}). As for EG, it is a greatest \emph{(post-)} fixpoint, so it could have been defined with a $\nu$ operator, where $\nu f = \neg\mu(\neg f) $. More precisely: \[EG(a,b)=\nu.y(a\wedge\diamond\mu.x((b\wedge y)\vee(a\wedge\diamond x)).\] This makes apparent a nesting of a $\nu$ and a $\mu$ operator. This is what makes the logic hard to simplify, compared to LTL which only contains one such fixpoint operator.  
\begin{definition}\label{interp_form_algebra}
    For any finite set of propositional variables $\bar{p}$ and any $CTL^f$-algebra $\A$, we can define a \emph{valuation} $V:\bar{p}\to A$. Then any $CTL^f$-formula $\varphi$ with variables in $\bar{p}$ can be interpreted as a term in $A$ (the \emph{interpretation} of $\varphi$ under $V$, $\varphi^A(V)$).

    Then, we can say the equality $\varphi(\bar{p})=\psi(\bar{p})$ is \emph{valid} if it interprets to $\top$ under every valuation to every $\A$.

    $\varphi$ and $\psi $ are said to be \emph{equivalent} if the equation $\varphi = \psi$ is valid. $\varphi$ is called a \emph{tautology} if it is equivalent to $\top$, and is said to be \emph{consistent} if it is not equivalent to $\bot$.

    Finally, we say $\varphi$ \emph{entails} $\psi$ if $\neg \varphi \vee \psi$ is a tautology, which we will note $\varphi \vdash \psi$ or $\varphi \leq \psi$.
\end{definition}

\subsection*{Semantics of fair CTL}\label{subsec:sem_CTLf}
\begin{definition}\label{forcing_rel_CTLf}
    We first define the notion of \emph{transition system}, ie. a pair $(S,R)$, where $S$ is a set, and $R$ a binary relation on $S$. Then, a \emph{$R$-path} is a (possibly infinite) sequence of nodes in $S$ such that $s_i R s_{i+1}$ for all $i$. Finally, for a finite set $\bar{p}$ of variables, define a \emph{$\bar{p}$-coloring} $\sigma : S \to \mathcal{P}(\bar{p})$.

    We can define the forcing relation $\models$ between nodes in $S$ and formulas $\varphi$ with variables in $\bar{p}$, by induction on $\varphi$:
    \begin{itemize}
        \setlength\itemsep{0em}
        \item[-] $s\not\models \bot$;
        \item[-] $s \models p$ if $p \in \sigma(s)$;
        \item[-] $s \models \neg\varphi$ if not $s \models \varphi$;
        \item[-] $s \models \varphi \vee \psi$ if $s\models \varphi$ or if $s\models \psi$;
        \item[-] $s\models \diamond\varphi$ if there exists $s'$ such that $sRs'$ and $s'\models \varphi$;
        \item[-] $s\models EU(\varphi,\psi)$ if there exists $n\geq 0$ and a path $s_0Rs_1R...Rs_n$ ($s_0=s$) such that $s_k\models\psi$ for every $k<n$ and $s_n\models\varphi$;
        \item[-] $s\models EG(\varphi,\psi)$ if there exists an infinite path $s_0Rs_1R...$ ($s_0=s$) such that $s_k\models\varphi$ for every $k$ and $s_j\models\psi$ infinitely often on the path.       
    \end{itemize}

\end{definition}

Remark that as a consequence, $s\models\Box\varphi$ if and only if for every successor $s'$ of $s$ by $R$, $s'\models\varphi$ holds.
Also, $s\models AR(\varphi,\psi)$ if and only if for every $n\geq 0$ and every path $s=s_0Rs_1R...Rs_n$, either $s_k\models\psi$ for some $k<n$, or $s_n\models\varphi$.

Similarly, $s\models AF(\varphi,\psi)$ if and only if for all infinite paths $(s)_k$ starting from $s_0$, if $s_j\not\models\psi$ infinitely often, then $s_k\models\varphi$ for some $k$.

Also, as a convention, we will consider the transition systems to be \emph{serial}, ie. that every node has a successor. Syntactically, this is represented by the axiom $\diamond\top = \top$.

Now, in order to prove completeness, we introduce the notion of \emph{complex algebra} of a transition system. This can be seen as a canonical algebra for the semantics of our logic. Then, we will be able to say that a formula holds semantically if, and only if, it can be derived in the complex algebra.

\begin{definition}\label{complex_algebra}
    Given a transition system $(S,R)$, its \emph{complex algebra} is the tuple \[\mathbb{P}_{(S,R)}=(\mathcal{P}(S),\emptyset,S\backslash(-),\cup,\diamond_R,EU_R,EG_R)\] which is the classical Boolean algebra over the powerset of $S$, along with operators \[\diamond_R(a)=R^{-1}[a]=\{s\in S \mid \exists t\in a, sRt \}\] and $EU_R$ and $EG_R$ defined as fixpoints (see above) so that we finally have a $CTL^f$-algebra. 
\end{definition}

\begin{lemma}\label{EU_as_union}
        For every $a,b$ in $\mathbb{P}_{(S,R)}$ it holds that \[EU_R(a,b)=\bigcup_{i=0}^{\infty}D_n(a,b)\] where $D_n$ is defined inductively as
        \begin{equation*}
            \begin{cases}
                D_0(a,b)=\emptyset\\
                D_{n+1}(a,b)=a\cup(b\cap\diamond_R D_n(a,b))
            \end{cases}
        \end{equation*}
        
\end{lemma}

\begin{proof}
    \begin{itemize}
        \item[$\supseteq$] By induction on n we show $D_n(a,b) \subseteq EU_R(a,b)$.\\ For $n=0$ this is obvious. For the induction step, let $s\in D_{n+1}(a,b)$. By the pre-fixpoint definition of $EU$ we have $a\leq EU_R(a,b)$ and $(b\cap\diamond EU_R(a,b))\leq EU_R(a,b)$. It follows that if $s$ is in $a$, it is directly in $EU_R(a,b)$; and in the other case, there exists $s'$ in $D_n(a,b)$ such that $sRs'$ By hypothesis $s'\in EU_R(a,b)$. We have exactly $s\in b\cap\diamond EU_R(a,b)$, from which we conclude $s\in EU_R(a,b)$.
        \item[$\subseteq$] We will show that $D(a,b) := \bigcup_{i=1}^{\infty}D_n(a,b)$ is a pre-fixpoint of $x \mapsto a \cup (b \cap \diamond x)$. From the definition of $EU$ the inclusion will directly follow.\\ Let $s\in a \cup (b \cap \diamond D(a,b))$. Note that $D_1(a,b)=a$ so if $s\in a$ it is in $D(a,b)$. And if $s\in (b \cap \diamond D_k(a,b))$ for some $k$ ($\diamond$ preserves countable unions), it is in $D_{k+1}(a,b)$ by definition. We can conclude here.
    \end{itemize}
\end{proof}
Note that $D_n(a,b)$ is the set: \[\{s\in S\mid \exists s_1,...s_n, s=s_0Rs_1R...Rs_n, s_k\in a \mbox{ for all } k<n \mbox{, and } s_n \in b\}\]
We can now state the corresponding lemma for $EG$:
\begin{lemma}\label{EG_as_set}
    For every $a,b$ in $\mathbb{P}_{(S,R)}$ it holds that $EG_R(a,b)$ is the set:
\begin{multline*}
    \{s\in S\mid \mbox{there exists an infinite path } s=s_0Rs_1R...\\
    \mbox{ such that }\forall k\in \mathbb{N} \mbox{ } s_k\in a\mbox{, and } s_{j}\in b \mbox{ for infinitely many } j \in\mathbb{N}\}
\end{multline*}
\end{lemma}
\begin{proof}
    \begin{itemize}
        \item[$\subseteq$] By induction on n we show $D'_n(a,b) \supseteq EG_R(a,b)$.\\ For $n=0$ this is obvious. For the induction step, let $s\in D_{n+1}(a,b)$. By the pre-fixpoint definition of $EU$ we have $a\leq EU_R(a,b)$ and $(b\cap\diamond EU_R(a,b))\leq EU_R(a,b)$. It follows that if $s$ is in $a$, it is directly in $EU_R(a,b)$; and in the other case, there exists $s'$ in $D_n(a,b)$ such that $sRs'$ By hypothesis $s'\in EU_R(a,b)$. We have exactly $s\in b\cap\diamond EU_R(a,b)$, from which we conclude $s\in EU_R(a,b)$.
        \item[$\supseteq$] We will show that $D(a,b) := \bigcup_{i=1}^{\infty}D_n(a,b)$ is a pre-fixpoint of $x \mapsto a \cup (b \cap \diamond x)$. From the definition of $EU$ the inclusion will directly follow.\\ Let $s\in a \cup (b \cap \diamond D(a,b))$. Note that $D_1(a,b)=a$ so if $s\in a$ it is in $D(a,b)$. And if $s\in (b \cap \diamond D_k(a,b))$ for some $k$ ($\diamond$ preserves countable unions), it is in $D_{k+1}(a,b)$ by definition. We can conclude here.
    \end{itemize}
\end{proof}

Operator $EU$ is \emph{constructive} (see \cite{Santo08}), because it can be described using pre-existing operators of the syntax; whereas the concept of `infinite path' introduced by operator $EG$ cannot be reduced to anything else, so $EG$ is not constructive. In particular, we do \emph{not} have the equality : \[EG_R(a,b)=\bigcap_{i=1}^\infty D'_n(a,b)\] where
\begin{equation*}
    \begin{cases}
        D'_0(a,b)=\mathcal{P}(S)\\
        D'_{n+1}(a,b)=a\cap\diamond_R EU_R(b\cap D'_n(a,b),a)
    \end{cases}
\end{equation*}
Although the analogy is pretty clear (the sets are defined as an sequence induced by the `fixpoint function' of the operator), it doesn't work because it fails to grasp infinite paths. For example, we can construct a tree belonging to every $D'_n(a,b)$ (for every $n$ there is a finite path where every node is in $a$, and $b$ contains at least $n$ nodes), but such that no infinite path has the wanted property (of belonging to the set described in \ref{EG_as_set}). Informally, the following infinite binary tree:

\begin{center} 
\begin{tikzpicture}
        \node {$\neg b$}[sibling distance = 2.5cm][level distance = 1 cm]
            child {node [red]{$b$}[sibling distance = 1cm]
                child{node {$\neg b$}[sibling distance = 1cm]}
                child{node {$\neg b$}[sibling distance = 2.5 cm]}}
            child {node {$\neg b$} [sibling distance = 2.5cm]
                child{node[red]{$b$}[sibling distance = 1cm]
                    child {node [red]{$b$}[sibling distance = 1cm]
                        child {node {$\neg b$}[sibling distance = 1cm]}
                        child {node {$\neg b$}[sibling distance = 2.5cm]}}
                    child {node {$\neg b$}[sibling distance = 2.5cm]}}
                child{node {$\neg b$}[sibling distance = 2.5cm]
                    child {node [red]{$b$}[sibling distance = 1cm]
                        child{node [red]{$b$}[sibling distance = 1cm]
                            child {node [red]{$b$}[sibling distance = 1cm]
                                child {node {$\neg b$}[sibling distance = 1cm]}
                                child {node {$\neg b$}[sibling distance = 2.5cm]}}
                            child {node {$\neg b$}[sibling distance = 2.5cm]}}
                        child {node {$\neg b$}[sibling distance = 2.5cm]}}
                    child {node {$\neg b$}[sibling distance = 2.5cm]
                        child{node{...}}
                        child{node{...}}}}};
    \end{tikzpicture}
\end{center}
is in every $D'_n(\mathcal{P}(S),b)$ but not in $EG_R(\mathcal{P}(S),b)$.

\begin{proposition}\label{equiv_complex_alg}
    Consider a transition system $(S,R)$ along with a a $\bar{p}$-coloring $\sigma : S \to \mathcal{P}(\bar{p})$. This coloring corresponds exactly to a valuation $V_\sigma:\bar{p}\to \mathcal{P}(S)$, as we in fact have a bijection between $\bar{p}$-colorings and such valuations.

    Then, for every $CTL^f$-formula $\varphi$ with variables in $\bar{p}$ and for every $s\in S$, we have: \[s\models\varphi \Longleftrightarrow s\in\varphi^{\mathbb{P}_{(S,R)}}(V_\sigma(\bar{p}))\]
\end{proposition}

\begin{proof}
    First, we show there is a bijection between $\bar{p}$-colorings $\sigma:S \to \mathcal{P}(\bar{p})$ and valuations $V:\bar{p}\to \mathcal{P}(S)$. Given $\sigma$, define $V_\sigma: p\mapsto\{s\in S \mid p\in \sigma(s)\}$. And conversely, given $V$, we can define $\sigma_V: v\mapsto\{p\in\bar{p}\mid v\in V(p)\}$.

    Second, we show the equivalence, by induction on the structure of $\varphi$ Looking at the definitions, it should be obvious for all cases up to diamond; as for $EU$ and $EG$, the wanted equivalence is a consequence of lemmas \ref{EU_as_union} and \ref{EG_as_set}.
\end{proof}
\section{Completeness proof}\label{sec:compl}    

This section will be dedicated to showing completeness of the above axiomatization of $CTL^f$. Equivalently, we show that consistent formulas in an algebra (not equal to $\bot$) can be satisfied semantically.

\begin{definition}\label{tree}
    A \emph{tree} is a transition system $(S,R)$ with a \emph{root}, such that for every $s$ in $S$, there is a unique path from the root to $S$. This implies a partial order $\preceq$, the reflexive-transitive closure of $R$, which has the property that $v\preceq v'$ if and only if $v$ lies on the unique path from the root to $v'$.
\end{definition}

\begin{theorem}\label{completeness}
    For every \emph{consistent} $CTL^f$-formula $\varphi_0(\bar{p})$, there exists a $\bar{p}$-colored tree with root $s_0$, such that $s_0 \models \varphi_0$.
\end{theorem}

The rest of this section consists of the proof of theorem \ref{completeness}. Let us fix a consistent formula $\varphi_0({\bar{p}})$. We then have a $CTL^f$-algebra $\A$ and a valuation $V:\bar{p}\to \A$ such that $\varphi_0^\A(V(\bar{p}))\neq\bot$. Now, since $\A$, $\bar{p}$ and $V$ are fixed in the rest of the section, we may leave them implicit and assimilate $\varphi_0$ to its interpretation.

\subsection*{Preliminaries}

We begin by proving a few facts about $EU$, $EG$, etc. that will be useful in the rest of the section.
\begin{proposition}\label{basic_prop_EU}
    For every $a,a',b,b',c$ in $\A$, the following hold:
    \begin{enumerate}
        \item $EU(a\vee a',b)=EU(a,b)\vee EU(a',b)$;
        \item $AR(a\wedge a',b)=AR(a,b)\wedge AR(a',b)$;
        \item $a\leq a'$ and $b\leq b'$ implies $EG(a,b)\leq EG(a',b')$;
        \item $a\leq a'$ and $b\leq b'$ implies $AF(a,b)\leq AF(a',b')$
        \item $AR(a,b)\wedge \neg b \leq \Box AR(a,b) $
        \item if $a\wedge c\leq\Box c $ and $b\wedge c\leq\Box c$, then $EU(a,b)\wedge c\leq EU(a\wedge\Box c,b\wedge\Box c)$.
    \end{enumerate}
\end{proposition}
\begin{proof}
    \begin{enumerate}
        \item $EU$ is defined as the fixpoint of a monotone function, therefore it is itself monotone. So we have $EU(a,b)\leq EU(a\vee a',b)$ and $EU(a',b)\leq EU(a\vee a',b)$, which entails $EU(a,b)\vee EU(a',b)\leq EU(a\vee a',b)$.
        For the other way, it suffices to prove that $EU(a,b)\vee EU(a',b)$ is a pre-fixpoint of the same function as $EU(a\vee a',b)$. We have, by monotonicity of $\diamond$ and by definition of $EU$, \[(a\vee a')\vee(b\wedge\diamond(EU(a,b)\vee EU(a',b)))\]\[=(a\vee(b\wedge\diamond(EU(a,b))))\vee (a'\vee(b\wedge\diamond(EU(a',b))))\]\[\leq EU(a,b)\vee EU(a',b)\]This proves our point.
        \item this is clear from the fact that $AR$ is the De Morgan dual of $EU$.
        \item clear from the definition of $EG$ as the fixpoint of a monotone function.
        \item idem.
        \item by definition of $AR$, we have $AR(a,b)\leq b\vee\Box(AR(a,b))$, which is equivalent to the statement we want to prove.
        \item suppose that $a\wedge c \leq \Box c$ and $b\wedge c \leq \Box c$. Let us prove that $d:=EU(a\wedge\Box c,b\wedge\Box c)\vee\neg c$ is a pre-fixpoint of the same function as $EU(a,b)$, which will entail our point by minimality of $EU(a,b)$.
        \begin{itemize}
            \item $a\wedge c\leq a$, and by assumption $a\wedge c\leq\Box c$, so $a\wedge c\leq a\wedge\Box c\leq (a\wedge\Box c)\vee(b\wedge\Box c\wedge\diamond EU(a\wedge\Box c,b\wedge\Box c))\leq EU(a\wedge\Box c,b\wedge\Box c)$;
            \item note that $b\wedge c\wedge \diamond d = (b\wedge c\wedge\diamond(\neg c))\vee((b\wedge c)\wedge\diamond EU(a\wedge\Box c,b\wedge\Box c))$; but $b\wedge c\wedge\diamond(\neg c)\leq b\wedge\Box c\wedge\diamond\neg c = \bot$. Then we have $b\wedge c\diamond d\leq (b\wedge\Box c)\wedge\diamond EU(a\wedge\Box c,b\wedge\Box c)\leq EU(a\wedge\Box c,b\wedge\Box c)$.
            \item From the two above points we conclude that $a\vee(b\wedge\diamond d)\leq d$. 
        \end{itemize}
    \end{enumerate}
\end{proof}

\begin{definition}\label{contextual_op}
    We define ternary \emph{contextual operations} $EU_c$ and $AF_c$:
    \begin{itemize}
        \item $EU_c(p,q,r)=p\vee(q\wedge\diamond EU(p\wedge r,q\wedge r))$,
        \item $AF_c(p,q,r)=AF(p,q)\wedge(p\vee\Box AR(q\wedge r,p))$.
    \end{itemize}
\end{definition}

\begin{proposition}\label{contextual_as_fixpoints}
    For every $p,q,r$ in $\A$, we have:
    \begin{enumerate}
        \item $EU_c(p,q,r)$ is the least pre-fixpoint of the monotone fuction $x\mapsto p\vee(q\wedge \diamond(r\wedge x))$;
        \item $AF_c(p,q,r)$ is the greatest post-fixpoint of the monotone function $x\mapsto p\vee\Box AR(q\vee(r\wedge x),p)$.
    \end{enumerate}
    
\end{proposition}

\begin{proof}
    \begin{enumerate}
        \item First, we have the equality \[EU_c(p,q,r)\wedge r = (p\wedge r)\vee(q\wedge r\wedge\diamond EU(p\wedge r,q\wedge r))\]\[=EU(p\wedge r,q\wedge r)\] by definition of $EU$. So $EU_c(p,q,r)$ is a fixpoint of the required function, as by definition: \[EU_c(p,q,r)=(\diamond(EU(p\wedge r,q\wedge r))\wedge q))\vee p\] Now, consider any fixpoint $s$ of this function. Then, $(p\wedge r)\vee(q\wedge r\wedge\diamond(r\wedge s))= r\wedge(p\vee(q\wedge\diamond(r\wedge s)))\leq r\wedge s$. By minimality, it entails that $EU(p\wedge r,q\wedge r)\leq r\wedge s$. Finally $EU_c(p,q,r)=p\vee(q\wedge\diamond EU(p\wedge r,q\wedge r))\leq p\vee(q\wedge\diamond(r\wedge s))\leq s$, which ends the proof.
        \item Consider function $\sigma :x\mapsto p\vee\Box AR(q\vee(r\wedge x),p)$.
        \begin{itemize}
            \item $AF_c(p,q,r)$ is a pre-fixpoint of $\sigma$.
            
            Note, by definition, that $AF_c(p,q,r)\leq AF(p,q)$. So \[p\vee\Box AR(q\vee AF_c(p,q,r),p)\]\[\leq p\vee\Box AR(q\vee AF(p,q),p)=AF(p,q)\] by definition of $AF$. By \ref{basic_prop_EU}(2), we have\[\sigma(AF_c(p,q,r))=(p\vee\Box AR(q\vee r,p))\wedge(p\vee\Box AR(q\vee AF_c(p,q,r)),p)\]\[\leq(p\vee\Box AR(q\vee r,p))\wedge AF(p,q)\]\[\leq AF_c(p,q,r).\]
            \item $AF_c(p,q,r)$ is the least such pre-fixpoint. Let $s\in A$ be such that $\sigma(s)\leq s$. Write $\alpha=AR(a\vee r,p)$. Then the following claim is true: \[AF(p\vee\diamond\neg \alpha, q\vee s)\wedge\Box\alpha\leq s\]
            Indeed, by hypothesis $p\leq s$ and $\Box AR(q\vee(r\wedge s),p)\leq s$. From there, $\neg q\wedge \neg s\wedge \alpha \leq \neg p \wedge \alpha\leq\Box\alpha$ by \ref{basic_prop_EU}(5). Applying \ref{basic_prop_EU}(6), we get that \[\neg AR(q\vee s,p)\wedge\alpha=EU(\neg q\wedge\neg s,\neg p)\wedge\alpha\]\[\leq EU(\neg q\wedge\neg s\wedge\Box\alpha,\neg p\wedge\Box\alpha).\] Now, negation inverts the inequation: \[AR(q\vee s\vee \diamond\neg\alpha,p\vee\diamond\neg\alpha)\leq AR(q\vee s,p)\vee\neg\alpha,\] to which we can apply $\Box$ on both sides. Now, $\neg\alpha=\neg AR(q\vee r,p)$ by definition; and \ref{basic_prop_EU}(2) tells us that \[AR(q\vee(r\wedge s),p)=AR(q\vee r,p)\wedge AR(q\vee s,p).\] The two combined give that \[\Box (AR(q\vee s,p)\wedge \neg\alpha)=\Box(AR(a\vee (r\wedge s),p)\vee\neg\alpha)\]\[\leq \Box AR(a\vee (r\wedge s),p)\vee\diamond\neg\alpha,\] by a property of modal algebras. We can combine all of the above and get that\[\Box AR(q\vee s\vee \diamond\neg\alpha,p\vee\diamond\neg\alpha)\leq \Box AR(a\vee (r\wedge s),p)\vee\diamond\neg\alpha\]\[\leq s\vee\diamond\neg\alpha.\] Remember also that $p\leq s$. This gives that \[(p\vee\diamond\neg\alpha)\vee\Box AR(q\vee s\vee \diamond\neg\alpha,p\vee\diamond\neg\alpha)\leq s\vee\diamond\neg\alpha.\] Apply definition of $AF$ and get: \[AF(p\vee\diamond\neg\alpha,q\vee s)\leq s\vee\diamond\neg\alpha,\] which is equivalent to our very first claim. In the end, by definition we have \[AF_c(p,q,r)=AF(p,q)\wedge(p\vee\Box\alpha)\]\[=(AF(p,q)\wedge p)\vee(AF(p,q)\wedge\Box\alpha)\]\[\leq p\vee (AF(p,q)\wedge\Box\alpha)\]\[\leq p\vee (AF(p\vee\diamond\neg\alpha,q\vee s)\wedge\Box\alpha)\mbox{ by monotonicity of }AF\]\[\leq s \mbox{ by everything above.}\] This is what we wanted to prove.
        \end{itemize}
    \end{enumerate}
\end{proof}

Note that replacing $EU$ by $EU_c$ and $AF$ bu $AF_c$ does not change the expressivity of the syntax, as $EU(a,b)=EU_c(a,b,\top)$ and $AF(a,b)=AF_c(a,b,\top)$. We then use notations $EU$ and $AF$ for either, as this presents no ambiguity.
\begin{proposition}\label{context_rule_gamma}
    For every $p,q,r,\gamma$ in $\A$, we have:
    \begin{enumerate}
        \item if $\gamma \wedge EU(p,q,r) \neq \bot$, then $\gamma \wedge EU(p,q,r\wedge\neg\gamma)\neq\bot$;
        \item if $\gamma \wedge AF(p,q,r) \neq \bot$, then $\gamma \wedge AF(p,q,r\wedge\neg\gamma)\neq\bot$.
    \end{enumerate}
\end{proposition}
\begin{proof}
    Let $p$, $q$, $r$ and $\gamma$ be elements of $A$. We can write $\delta = \neg\gamma$ and prove the contraposition, namely\[EU(p,q,r\wedge \delta)\leq \delta\Rightarrow EU(p,q,r)\leq \delta.\] Suppose that $EU(p,q,r\wedge \delta)\leq \delta$, ie. $EU(p,q,r\wedge \delta)=EU(p,q,r\wedge \delta)\wedge\delta$. By \ref{contextual_as_fixpoints}, writing $\sigma: (p,q,x)\mapsto p\vee(q\wedge\diamond x) $ we have:\[\sigma(p,q,r\wedge EU(p,q,r\wedge \delta))=\sigma(p,q,r\wedge EU(p,q,r\wedge \delta)\wedge\delta)\]\[=EU(p,q,r\wedge \delta).\]But then, by minimality, we get that \[EU(p,q,r)\leq EU(p,q,r\wedge \delta)\leq\delta\mbox{ by assumption.}\]
    
    The proof for $AF$ is strictly analoguous. The above proof actually makes use of a general fixpoint argument.
\end{proof}

The last property is not obvious semantically, and explaining why it holds helps understand what it means. Semantically, the hypothesis is : $\gamma\cap EU_R(p,q,r)\not=\emptyset$. By replacing $EU$ by its expression given in \ref{EU_as_union}, and by basic rules of the algebra, we get that this is equivalent to \[(\gamma\cap p)\cup \bigcup_{n=0}^\infty(q\cap\gamma\cap\diamond_RD_n(p\cap r,q\cap r))\not=\emptyset.\] In other terms, we suppose that:
\begin{equation*}
    \begin{cases}
        \gamma\cap p \not=\emptyset \\
        \mbox{or }\exists n\geq 0,\exists s\in q\cap\gamma, sRs' \mbox{ for } s' \mbox{ in }D_n(p\cap r,q\cap r).
    \end{cases}
\end{equation*}
Now remember that we want to prove:
\begin{equation*}
    \begin{cases}
        \gamma\cap p \not=\emptyset \\
        \mbox{or }\exists n\geq 0,\exists s\in q\cap\gamma, sRs' \mbox{ for } s' \mbox{ in }D_n(p\cap r\cap\neg\gamma,q\cap r\cap\neg\gamma).
    \end{cases}
\end{equation*}
The case where $\gamma\cap p \not=\emptyset$ is obvious. In the other case, take $n\geq 0$, $s\in q\cap\gamma$ and $s'\in D_n(p\cap r,q\cap r)$, where $sRs'$. We then have a path $s'=s'_0R...Rs'_n$, where $s'_k\in q\cap r$ $\forall k<n$ and $s'_n\in p\cap r$. Now, if $s'_n$ is in $\gamma$, then we have $\gamma\cap p \not= \emptyset$, which was wanted. If not, then $s'_n\in \neg\gamma \cap r\cap p$. Then we may discriminate on whether $s'_{n-1}$ belongs or not to $\gamma$. If it does not, then $s'_{n-1}\in \neg\gamma\cap r\cap q$; we go back inductively until we get a $k$ such that $s'_k$ is in $\gamma$. If this occurs for no $k\geq 0$, then $s'_{-1}:=s$ is a predecessor of $s'_0$ belonging to $\gamma\cap q$ by hypothesis. In either case, we get a node $s_k$ in $q\cap\gamma$, such that its successor is in $D_{n-k}(p\cap r\cap\neg\gamma,q\cap r \cap\neg\gamma)$.
\begin{lemma}\label{another_fix_AFc}
    For any elements $p$, $q$, $r$ in a $CTL^f$-algebra $\A$, we have \[AF_c(p,q,r)=p\vee\Box((q\vee r)\wedge AF_c(p,q,r)).\]
\end{lemma}
\begin{proof}
    Write $t:=q\vee(r\wedge AF_c(p,q,r)$. By proposition \ref{contextual_as_fixpoints}, $AF_c(p,q,r)=p\vee\Box AT(t,p)$. Using this fact and the expression of $AR$ as a fixpoint, we have \[AR(t,p)=t\wedge(p\vee\Box AR(t,p))=t\wedge AF_c(p,q,r)=(q\vee r)\wedge AF_c(p,q,r).\] Therefore, we have the wanted equality.
\end{proof}
\begin{definition}\label{NNF}
    A $CTL^f$-formula is in \emph{negative normal form (NNF)} if it is constructed by applying operators ($\wedge,\vee,\Box,\diamond,EU_c,AF_c,AR,EG$) to literals, ie. formulas of the form $\bot\mid\top\mid p\mid \neg p$.
\end{definition} 
\begin{proposition}\label{equiv_NNF}
    Every $CTL^f$-formula is equivalent to one that is in NNF.
\end{proposition}
\begin{proof}
    Negations can be pushed inwards, thanks to dualities, eg. $\neg\Box\equiv\diamond\neg$, etc.
\end{proof}

This is now that we make use of the results of (J\'onsson-Tarski) duality on modal algebras, shown for example in \cite{GehvG22} (chapter 4).
\begin{definition}\label{dual_frame}
    Given a modal algebra $\A$, we define its \emph{dual frame} as the pair $\A_*=(A_*,R_*)$, where
    \begin{itemize}
        \item $A_*$ is the set of \emph{ultrafilters} of $\A$ (given that $\A$ is a Boolean algebra, its filters are all incomparable for the inclusion order; they correspond to upward-closures of the atoms of $\A$);
        \item $R_*$ is the relation on $A_*$ induced by $\diamond$, so that $xR_*y$ if and only if $\diamond [y]\subseteq x$.
    \end{itemize}
\end{definition}
Still from \cite{GehvG22}, we can state the following theorem:
\begin{theorem}\label{dual_modal_algebra}
    Any modal algebra $\A$ is isomorphic to the complex algebra of its dual frame.
\end{theorem}
The difficulty here is that this is not true for every $CTL^f$-algebra.
\begin{lemma}\label{exists_succ_R}
    Let $\A$ and $\A_*$ be a modal algebra and its dual frame. Let $x\in A_*$. Consider $a\in\A$ such that $\diamond a\in x$. Then there exists $y\in A_*$ such that $a\in y$ and $xR_*y$.
\end{lemma}
\begin{proof}
    Consider the set $\{b\in A\mid \diamond b\notin x\}$. It is an ideal in $\A$ not containing $a$. Indeed, it is downwards-closed, because if $c\leq b$, and $\diamond b \notin x$, then $\diamond c \notin x$ by monotonicity of $\diamond$. And it is stable by join, as $\diamond$ preserves join, so if it contained $\diamond(b\vee c)$ it would contain $\diamond b$. Therefore its complementary is a filter $:=y$ containing $a$, and $xR_*y$ quite obviously from definition.
\end{proof}
\begin{definition}\label{rho_equiv}
    Let $\rho$ be a finite a set of formulas. For any $x,x'\in \A_*$, define \[x\sim_\rho x' \Leftrightarrow x\cap\rho=x'\cap\rho\] We call the equivalence class of a point $x$ under $\sim_\rho$ the \emph{$\rho$-type of $x$}. For any $x\in \A_*$, define the \emph{characteristic formula} \[\kappa(x,\rho):=\bigwedge\{\gamma\mid\gamma\in\rho\cap x\}\wedge\bigwedge\{\neg\gamma\mid\gamma\in\rho\backslash x\}\]
\end{definition}
\begin{lemma}\label{equiv_as_kappa}
    For any set of formulas $\rho$ and points $x,x'\in\A_*$, we have \[x\sim_\rho x' \Leftrightarrow \kappa(x,\rho)\in x'.\] 
\end{lemma}
\begin{proof}
    $x'$ being an ultrafilter, we have $\kappa\in x'$ if and only if for all $\gamma\in x\cap\rho$, $\gamma\in x'$, and for all $\gamma\in \rho\backslash x$, $\gamma\notin x'$. This is another way of saying $x\cap\rho = x'\cap\rho$.
\end{proof}
We combine this lemma with proposition \ref{context_rule_gamma} to obtain the following fact, which will allow to make `jumps' in the ultrafilter frame of $\A$.
\begin{lemma}\label{add_neg_kappa}
    Let $\rho$ be a finite set of formulas, let $\heartsuit\in\{EU,AF\}$, and let $\varphi$, $\psi$, and $\chi$ be formulas. For any $x\in\A_*$ such that $\heartsuit(\varphi,\psi,\chi)\in x$, there exists $x'\in\A_*$ such that $x\sim_\rho x'$ and $\heartsuit(\varphi,\psi,\chi\wedge\neg\kappa(x,\rho))\in x'$.
\end{lemma}
\begin{proof}
    We have $\kappa(x,\rho)\wedge\heartsuit(\varphi,\psi,\chi)\in x$, so it is not equal to $\bot$. Then, by \ref{context_rule_gamma}, $\kappa(x,\rho)\wedge\heartsuit(\varphi,\psi,\chi\wedge\neg\kappa(x,\rho))\not=\bot$. Then it must belong to some ultrafilter $x'$ in $\A_*$, which is upwards-closed, so $\kappa(x,\rho)\in x'$ and $\heartsuit(\varphi,\psi,\chi\wedge\neg\kappa(x,\rho))\in x'$. Moreover, by \ref{equiv_as_kappa}, we get our equivalence.
\end{proof}
\begin{definition}\label{F-L closure}
    A set of $CTL^f$-formulas $\Gamma$ is called \emph{(Fischer-Ladner) closed} if the following hold:
    \begin{itemize}
        \setlength\itemsep{0em}
        \item $EU(\top,\top,\top)\in\Gamma$;
        \item if $\varphi$ is in $\Gamma$, then any subformula of $\varphi$ is in $\Gamma$ too;
        \item if $EG(\varphi,\psi)\in\Gamma$, then $\diamond EU(\psi\wedge EG(\varphi,\psi),\varphi)\in\Gamma$;
        \item if $AR(\varphi,\psi)\in\Gamma$, then $\Box AR(\varphi,\psi)\in\Gamma$;
        \item if $EU(\varphi,\psi,\chi)\in\Gamma$, then $\diamond(\chi\wedge EU(\varphi,\psi,\chi))\in\Gamma$;
        \item if $AF(\varphi,\psi,\chi)\in\Gamma$, then $\Box AR(\psi\vee\chi,\varphi)\in\Gamma$;
    \end{itemize}
    The \emph{closure} of a set of $CTL^f$-formulas is the smallest closed set containing it. 
\end{definition}
\begin{lemma}\label{finite_closure}
    The closure of a finite set of $CTL^f$-formulas is finite.
\end{lemma}
\begin{proof}
    Obvious from definition \ref{F-L closure}.
\end{proof}

\subsection*{Model construction}
Remember that we have fixed a consistent formula $\varphi_0$ at the beginning of this section (along with valuation $V:\bar{p}\to\A)$, and that we need to construct a model (a tree) in which it is satisfied. By \ref{finite_closure}, define $\Gamma_0$ to be the (finite) closure of $\{\varphi_0\}$.

\begin{definition}\label{eventuality_fml}
    We will call an \emph{eventuality formula} any formula of the form $\heartsuit(\varphi,\psi,\chi)$, with $\heartsuit$ being $AF$ or $EU$.

    Moreover, we note the set of eventuality formulas in propositional variables $\bar{p}$ as $\Evp$.
\end{definition}

The tree $S$ we will construct will be a union of finite trees, which use a notion of \emph{partial tableau for $\Gamma_0$ in $\A$}. Informally, a partial tableau will consist of a finite tree $T$, along with two labellings $\alpha$ and $\beta$. Labelling $\alpha$ assigns to each node of $T$ an ultrafilter of $\A$, which corresponds to the set of formulas we wish to force in the node. On the other end, $\beta$ records the `current status' of eventuality formulas in $\Gamma_0$, via a data structure consisting of a finite list of tuples of the form $(\theta,\sigma,\rho,\tilde{\chi})$ for each node $v$. Each tuple is made of an eventuality formula $\theta$ of $\Gamma_0$ which lies in $\alpha(v')$ for some $v'\preceq v$; a status $\sigma$ ($a$ for active, $f$ for frozen or $e$ for extinguished); a finite set of formulas $\rho$, called the \emph{relevance set}; and a \emph{context formula} $\tilde{\chi}$ which will be a strenghtening of $\chi$.

\begin{definition}\label{partial_tableau}
    Let $\Gamma_0$ be a finite, closed set of $CTL^f$-formulas with variables in $\bar{p}$. Define \[\Sigma:=(\Gamma_0\cap\Evp)\times\{a,f,e\}\times\mathcal{P}_f(CTL^f(\bar{p}))\times CTL^f(\bar{p}).\] A \emph{partial tableau for $\Gamma_0$ in $\A$} is a tuple $(T,\alpha,\beta)$, where
    \begin{itemize}
        \setlength\itemsep{0em}
        \item $T$ is a finite tree;
        \item $\alpha$ is a function from $T$ to $\A_*$, the set of ultrafilters of $\A$;
        \item $\beta$ is a function from $T$ to $\Sigma_*$, the set of finite words over $\Sigma$.
    \end{itemize}

    For each $v\in T$ we write $\ell(v)$ for the length of $\beta(v)$. 
    
    For each $1\leq k\leq \ell(v)$, we write $\beta(v)_k=(\theta(v)_k,\sigma(v)_k,\rho(v)_k,\tilde{\chi}_k)$, with $\theta(v)_k=\heartsuit(v)_k(\varphi(v)_k,\psi(v)_k,\chi(v)_k)$.
\end{definition}

We also need to impose a few conditions on partial tableaux so that they are `well-formed', namely:
\begin{enumerate}[label=(\alph*)]
    \setlength\itemsep{0em}
    \item elements of the list $\beta(v)$ persist in $\beta(v')$ for tree successors $v'$ of $v$;
    \item if the first coordinate $\varphi$ of an eventuality formula lies in $\alpha(v)$, then it is extinguished;
    \item $\rho$ always contains $\Gamma_0$; 
    \item $EU$-formulas are never frozen;
    \item $\tilde{\chi}$ is a strenghtening of $\chi$;
    \item eventuality formulas that occur earlier in $\beta$ always lie in the relevance set;
    \item $\alpha$ contains all non-extinguished eventuality formulas in $\beta$.
\end{enumerate}

Formally:
\begin{definition}\label{well_formed_partial_tableau}
    The partial tableau $(T,\alpha,\beta)$ for $\Gamma_0$ in $\A$ is said to be \emph{well-formed} if the following conditions hold for all $v\in T$ and for all $1\leq k\leq \ell(v)$:
    \begin{enumerate}[label=(\alph*)]
        \setlength\itemsep{0em}
        \item $\forall v'\succeq v$, $\ell(v')\geq\ell(v)$ and $\theta(v')_k = \theta(v)_k$;
        \item $\varphi(v)_k\in\alpha(v)\Rightarrow\sigma(v)_k=e$;
        \item $\Gamma_0\subseteq\rho(v)_k$
        \item $\heartsuit(v)_k=EU \Rightarrow \sigma(v)_k\not=f$;
        \item $\tilde{\chi}(v)_k\vdash\chi(v)_k$;
        \item $k'<k\Rightarrow\heartsuit_{k'}(\varphi(v)_{k'},\psi(v)_{k'},\tilde{\chi}(v)_{k'})\in\rho(v)_k$;
        \item $\sigma(v)_k\not=e\Rightarrow\heartsuit_{k}(\varphi(v)_{k},\psi(v)_{k},\tilde{\chi}(v)_{k})\in\alpha(v)_k$.
    \end{enumerate}
\end{definition}

Now, we will describe how to unravel a well-formed partial tableau. As we will prove, the unravelling of a well-formed partial tableau will remain well-formed. Then, we will construct a tree which statisfies $\varphi_0$, thanks to the unravelling and its preservation of properties.

First, we give an informal explanation of the unravelling. We consider the ultrafilter frame of $\A$, $\A_*=(A_*,R_*)$ (\ref{dual_frame}). Normally, in a tableau construction, we would add successors to $v$ for all $\diamond$-formulas in $\Gamma_0\cap\alpha(v)$, and label them by $R_*$-successors of $\alpha(v)$. But here, we also need to treat eventuality-formulas.

Instead of labelling children of $v$ by successors of $\alpha(v)$, we make a `jump' in the ultrafilter frame from $\alpha(v)$ to $x_v$, given by the first active eventuality formula $\heartsuit_m(\varphi_m,\psi_m,\chi_m)$ in the list $\beta(v)$; and we label children of $v$ by $R_*$-successors of $x_v$. The choice of $x_v$ is guided by the relevance set $\rho_m$, and ensures that:
\begin{itemize}
    \item[-]$\alpha(v)$ and $x_v$ have the same $\rho_m$-type;
    \item[-]$\neg\kappa(\alpha(v),\rho_m)$ can be added conjuctively to $\tilde{\chi}_m$ while keeping the partial tableau well-formed.  
\end{itemize}
The existence of such an $x_v$ will be given by lemma \ref{add_neg_kappa}. The point of making this jump is to reinforce the statement $\heartsuit_m(\varphi_m,\psi_m,\chi_m)$, so that no infinite loops occur in the construction.

\begin{definition}\label{one-step_unravelling}
    We define the \emph{one-step unravelling} of a well-formed partial tableau $(T,\alpha,\beta)$. For each leaf $v$ of $T$, we add a finite set of children $C_v:=\{w_\lambda\mid\diamond\lambda\in\Gamma_0\cap\alpha(v)\}$. Note that $C_v$ is not empty, because $\diamond(\top\wedge EU(\top,\top,\top))\in\Gamma_0$ by definition \ref{F-L closure}, and because $\diamond(\top\wedge EU(\top,\top,\top))=\top$ in $\A$, so it is in ultrafilter $\alpha(v)$. Now, we specify a value for $\alpha$ and $\beta$ on each of those children.

    Fix a leaf $v$. We choose an auxiliary filter $x_v$ in $\A_*$. If $\sigma_k\not=$ a for all $1\leq k \leq \ell(v)$, define $x_v=\alpha(v)$. Otherwise, name $m$ the first such $k$ with $\sigma_k=$ a. We call $m$ the \emph{active index} at $v$. Because $(T,\alpha,\beta)$ is well-formed, by \ref{well_formed_partial_tableau}(g), we have $\heartsuit_m(\varphi_m,\psi_m,\tilde{\chi}_m)\in\alpha(v)$. Therefore, by \ref{add_neg_kappa}, we have $\heartsuit_m(\varphi_m,\psi_m,\tilde{\chi}_m\wedge\neg\kappa(\alpha(v),\rho_m))\in x_v$ for some $x_v\sim_{\rho_m}\alpha(v)$. Write $\gamma_v:=\kappa(\alpha(v),\rho_m)$.
    
    Let $w_\lambda\in C_v$. We use $x_v$ to define $\alpha(w_\lambda)$ and $\beta(w_\lambda)$. For $\alpha$, there are two cases. Note that $m$ might not be defined, in which case we act as if $\heartsuit_m=AF$ for $\alpha$, and as if $m=\infty$ for $\beta$. (WHAT do we take for $rho_m$ if m is not def ??)
    \begin{itemize}
        \item If $\heartsuit_m=AF$, or $\heartsuit_m=EU$ and $\lambda\not=\chi_m\wedge EU(\varphi_m,\psi_m,\chi_m)$: since $\diamond\lambda\in\Gamma_0\cap\alpha(v)$, we have $\diamond\lambda\in x_v$, because $\Gamma_0\subseteq\rho_m$ and $\alpha(v)\sim_{\rho_m} x_v$. Therefore, by lemma \ref{exists_succ_R}, pick $\alpha(w_\lambda)$ such that $x_vR_*\alpha(w_\lambda)$ and $\lambda\in\alpha(w_\lambda)$.
        \item If $\heartsuit_m=EU$ and $\lambda=\chi_m\wedge EU(\varphi_m,\psi_m,\chi_m)$: by \ref{well_formed_partial_tableau}(b) and the fact that $\sigma_m=a$, we have $\varphi_m\notin\alpha(v)$. Since $\alpha(v)\sim_{\rho_m} x_v$ and $\varphi_m\in\Gamma_0\subseteq\rho_m$, we have $\varphi_m\notin x_v$, so $\neg\varphi_m\in x_v$. Also, $EU(\varphi_m,\psi_m,\tilde{\chi}_m\wedge\neg\gamma_v)\in x_v$ by the choice of $x_v$. Applying the general fact (\ref{contextual_as_fixpoints}) that $EU(p,q,r)\wedge \neg p\leq\diamond(r\wedge EU(p,q,r))$, we obtain $\diamond(\tilde{\chi}_m\wedge\neg\gamma_v\wedge EU(\varphi_m,\psi_m,\tilde{\chi}_m\wedge\neg\gamma_v))\in x_v$. By lemma \ref{exists_succ_R}, pick $\alpha(w_\lambda)$ such that $x_vR_*\alpha(w_\lambda)$ and $\tilde{\chi}_m\wedge\neg\gamma_v\wedge EU(\varphi_m,\psi_m,\tilde{\chi}_m\wedge\neg\gamma)\in\alpha(w_\lambda)$. Note that in particular, $\chi_m\wedge EU(\varphi_m,\psi_m,\chi_m)\in\alpha(w_\lambda)$, since $\tilde{\chi}_m\wedge\neg\gamma_v\leq\chi_m$, and $EU$ is monotone.
    \end{itemize}
    
    Then, we construct the word $\beta(w_\lambda)$ as an update of the word $\beta(v)$. We apply the following steps:
    \begin{enumerate}
        \item Let New($w_\lambda$) $:=\{\theta\in\alpha(w_\lambda)\cap\Gamma_0\cap Ev(\bar{p})\mid\forall 1\leq k\leq \ell(v):\mbox{ if }\theta_k(v)=\theta, \mbox{ then }\sigma_k(v)=\mbox{e}\}$. Note that this set is not empty, as it contains $EU(\top,\top,\top)$. For each $\theta=\heartsuit(\varphi,\psi,\chi)$ in New($w_\lambda$), add one letter, $(\theta,\mbox{a},\rho',\chi)$, to the end of the word, where $\rho'=\bigcup_{k=1}^{\ell(v)}\rho_k$.
        \item For each position $k$, put \\ $\tilde{\chi}(w_\lambda)_k=\begin{cases}
            \tilde{\chi}(v)_k \mbox{ if } k<m,\\
            \tilde{\chi}(v)_m\wedge\neg\gamma_v \mbox{ if } k=m,\\
            \chi(v)_k \mbox{ if } k>m\\
        \end{cases}$
        \item For each position $k>m$, add the formula $\heartsuit_m(\varphi_m,\psi_m,\tilde{\chi}(v)_m\wedge\neg\gamma_v)$ to the set $\rho_k$.
        \item For each positon $k$ such that $\varphi_k\in\alpha(w_\lambda)$, change $\sigma_k$ into e.
        \item For each position $k$, if $\theta_k = EU(\varphi_k,\psi_k,\chi_k)$ and $\lambda$ is not equal to $\chi_k\wedge EU(\varphi_k,\psi_k,\chi_k)$, change $\sigma_k$ into e.
        \item For each position $k$, if $\heartsuit_k = AF$, $\psi_k\in\alpha(w_\lambda)$, and $\sigma_k =$ a, change $\sigma_k$ into f.
        \item For each position $k<m$, if $\heartsuit_k = AF$, $\sigma_k=$ f, $\varphi_k\notin\alpha(w_\lambda)$ and $\psi_k\notin\alpha(w_\lambda)$, change $\sigma_k$ into a.
    \end{enumerate}
\end{definition}
\begin{lemma}\label{unravelling_wf}
    The one-step unravelling of a well-formed partial tableau is well-formed.
\end{lemma}
\begin{proof}
    Let us check that all conditions of definition \ref{well_formed_partial_tableau} are fulfilled.
    \begin{enumerate}[label=(\alph*)]
        \setlength\itemsep{0em}
        \item is natural from the fact that $\beta$ is contructed as an update of its predecessor;
        \item is verified, by point (4) of the construction of $\beta$;
        \item is verified by (1);
        \item by (1) and (5);
        \item by (1) and (2);
        \item by (1);
        \item Let $v$ be a leaf of $T$, $m$ the active index at $v$, and $w_\lambda$ a child of $v$ in the one-step unravelling of $T$. Let $1\leq k\leq\ell(v)$ be such that $\sigma(w_\lambda)_k\not=$ e. We need to show that $\heartsuit_k(\varphi_k,\psi_k,\tilde{\chi}(w_\lambda)_k)\in\alpha(w_\lambda)$. We distinguish three cases:
        \begin{itemize}
            \item If $k<m$: by definition of $m$, we have $\sigma(v)_k\not=$ a, but $\sigma(w_\lambda)_k\in\{\mbox{a,f}\}$. Therefore $\sigma(v)_k=$ f. In particular $\heartsuit_k=AF$. Because $T$ is well-formed, $AF(\varphi_k,\psi_k,\tilde{\chi}(v)_k)$ belongs to $\rho_m\cap\alpha(v)$, so $AF(\varphi_k,\psi_k,\tilde{\chi}(v)_k)$ is in $x_v$. Also, since $\sigma(v)\not=$ e, we have $\neg\varphi_k\in x_v$. From \ref{another_fix_AFc}, it follows that \[AF(\varphi_k,\psi_k,\tilde{\chi}(v)_k)\wedge\neg\varphi_k\leq\Box AF(\varphi_k,\psi_k,\tilde{\chi}(v)_k).\] Finally $AF(\varphi_k,\psi_k,\tilde{\chi}(w_\lambda)_k)=AF(\varphi_k,\psi_k,\tilde{\chi}(v)_k)$ is in $\alpha(w_\lambda)$, as $w_\lambda$ is a child of $v$ by construction.
            \item If $k=m$: if $\heartsuit_m = EU$, then $\sigma(w_\lambda)_m \not=e$ only if $\lambda$ is equal to $\chi_m\wedge EU(\varphi_m,\psi_m,\chi_m)$, and it is true by construction that $\alpha(w_\lambda)$ contains $EU(\varphi_m,\psi_m,\tilde{\chi}_m)$. If $\heartsuit_m = AF$, note that $x_v$ contains $AF(\varphi_m,\psi_m,\tilde{\chi}_m)$ by construction. Also, $\neg\varphi_m\in x_v$ because $\sigma(v)_m =$ a. Again using lemma \ref{another_fix_AFc}, $AF(\varphi_m,\psi_m,\tilde{\chi}_m)\in\alpha(w_\lambda)$.
            \item If $k>m$: If $\heartsuit_k = EU$, then, since $\sigma(w_\lambda)_k\not=$ e, by (5) in definition \ref{well_formed_partial_tableau} we must have $\lambda = \chi_k\wedge EU(\varphi_k,\psi_k,\chi_k)$. In particular $EU(\varphi_k,\psi_k,\chi_k)\in\alpha(w_\lambda)$. If $\heartsuit_k=AF$, then since $\tilde{\chi}(v)_k\leq\chi_k$ and $AF(\varphi_k,\psi_k,\tilde{\chi}(v)_k)\in\alpha(v)$ by well-formedness of $T$, we have $AF(\varphi_k,\psi_k,\chi_k)\in\alpha(v)$. Since $\sigma(v)_k\not=$ e, we have $\neg\varphi_k\in\alpha(v)$. Since $AF(\varphi_k,\psi_k,\chi_k)$ and $\neg\varphi_k$ lie in $\Gamma_0\subseteq\rho(v)_k$, we also have $AF(\varphi_k,\psi_k,\chi_k)$ and $\neg\varphi_k$ in $x_v$. As before, $\Box AF(\varphi_k,\psi_k,\chi_k)$ is in $x_v$, so $\alpha(w_\lambda)$ contains $AF(\varphi_k,\psi_k,\chi_k)$.
        \end{itemize}
    \end{enumerate}
\end{proof}
\begin{definition}\label{model_for_consistency}
    We define a tree which models $\varphi_0$.
\end{definition}

\bibliography{bibli}
\end{document}