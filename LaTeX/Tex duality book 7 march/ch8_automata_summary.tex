

\chapter{Automata Theory}\label{ch:AutThry}
In this chapter\footnote{The content of this chapter is omitted from this preprint version, but we give the reader an impression its contents.} we  discuss applications of duality theory to automata theory. We started our introduction to duality theory with the discrete duality for finite distributive lattices, cf. Section~\ref{sec:finDLduality}. The finite case came first chronologically and, while it is simpler than the full topological theory, it contains many of the central ideas of the subject in embryonic form. For applications of duality theory in the theory of regular languages, a similar r\^ole is played by the result that one may associate, to each finite state automaton, a finite monoid with a universal property. We start by showing that this foundational result of algebraic automata theory is a case of discrete duality for complete and atomic Boolean algebras with additional operations.

In the first section of this chapter, we consider regular languages one at a time, and we show that the dual of the residuation ideal generated by such a language is finite -- essentially by virtue of the simple fact that any finite state machine can only be equipped with finitely many choices of pairs $(I,F)$, of initial and final states. We place this result within discrete duality, and, as a consequence, we see that regular languages over an alphabet $A$ are precisely those subsets of $\cP(A^*)$ that are recognised by finite monoids. 

In Section~\ref{sec:freeprofmonoid}, we want to consider the set of all regular languages together as a whole. We will see that the set of all regular languages over $A$ form a Boolean residuation ideal in $\cP(A^*)$, albeit not a complete one. So we have to switch from the discrete duality to Stone's topological duality. We will be rewarded by seeing that its dual is a very natural object from topological algebra, namely, the \emph{free profinite monoid} over $A$. 

Next, we apply the subalgebra--quotient-space duality to the pair: regular languages over $A$ and its dual space, the free profinite monoid over $A$, focussing in on subalgebras that are residuation ideals. This will culminate in an account of the pairing of so called (pseudo-)varieties of regular languages and relatively free profinite monoids that is known in automata theory via the combination of \emph{Eilenberg's Theorem} and \emph{Reitermann's Theorem}. We will explain all this jargon in detail along the way. We will also give a few classical examples of decidability results provided by equational axiomatizations of such pseudo-varieties. 

Finally, in Section~\ref{sec:openmult}, we again apply the subalgebra--quotient-space duality to the pair: regular languages over $A$ and its dual space, the free profinite monoid over $A$, albeit this time we focus on subalgebras closed under concatenation. This points in the direction of \emph{categorical logic} and \emph{Lawvere's hyperdoctrines} rather than automata theory and is beyond the scope of this book. We just give a few elementary observation in the short final section.

\section{The syntactic monoid as a dual space}\label{sec:syntmon}


\section{Regular languages and free profinite monoids}\label{sec:freeprofmonoid}





\section{Equations, subalgebras, and profinite monoids}\label{sec:EilReittheory}





\section{Open multiplication}\label{sec:openmult}


