\chapter{Topology and order}\label{chap:TopOrd}

In this chapter we present some material at the interface of topology and order theory. The culminating results of the chapter provide an equivalence between certain compact Hausdorff topological spaces equipped with orders, which were first introduced by Nachbin, and certain non-Hausdorff spaces known as stably compact spaces. Nachbin's thesis of the same title as this chapter was first published in Portuguese in 1950 and later translated to English \cite{Nachbin64}. It is a nice text and we recommend it as supplemental reading.\endnote{For general references on topology we recommend the updated edition of Engelking \cite{Engelking89}. For a comprehensive reference on topology and order we recommend \cite{Getc80}.}

\section{Topological spaces}

We expect readers to be familiar with the basic definitions and notions of topology. Nevertheless, we give them here in order to fix notation and nomenclature. For readers who need further introduction to topology and for general topology references beyond what we have been able to include here, we recommend  a classical book on General Topology such as \cite{Engelking89}.

A \emphind{topological space} is a pair $(X,\tau)$ where $X$ is a non-empty set and $\tau$ is a bounded sublattice of $\cP(X)$ which is closed under arbitrary unions. The elements of $\tau$ are called \emphind{open sets} while their complements are said to be \emphind{closed sets}. We will often simply write $X$ for a topological space (if the collection of opens is clear). The collection of opens may then be denoted by $\Omega X$. The collection of closed subsets of $X$ is denoted $\cC X$. A subset $K \subseteq X$ is called \emphind{clopen} if it is both closed and open. The collection of clopen subsets of a space is denoted $\Clp(X)$.

A map $f\colon X\to Y$ between topological spaces is \emphind{continuous} provided the preimage map $f^{-1}\colon\cP(Y)\to\cP(X)$ maps opens of $Y$ to opens of $X$. That is, there is a restriction of $f^{-1}$ to $\Omega(Y)$ which makes the following diagram commute:
\[
\begin{tikzpicture}
\node(PY) at (0,1.5) {$\cP(Y)$};
\node(PX) at (3,1.5) {$\cP(X)$};
\node(OY) at (0,0) {$\Omega(Y)$};
\node(OX) at (3,0) {$\Omega(X)$};

\draw[->,>=stealth'] (PY)to node[above] {$f^{-1}$} (PX);
\draw[->,dashed,>=stealth'] (OY) to node[below] {} (OX);
\draw[right hook->,>=stealth'] (OY) to (PY);
\draw[right hook->,>=stealth'] (OX) to (PX);

\end{tikzpicture}
\]

\begin{example}
Let $X$ be a set, then $\tau=\cP(X)$ is a topology on $X$.  This topology is known as the \emphind{discrete topology} on $X$. Notice that any mapping from a discrete space to any topological space is continuous.
\end{example}

\begin{example}
Let $X$ be a set, then $\tau=\{\emptyset,X\}$  is a topology on $X$.  This topology is known as the \emphind{indiscrete topology} on $X$. Notice that all mappings to an indiscrete space from any topological space are continuous.
\end{example}

\begin{example}\label{exp:real}
Let $\R$ be the set of real numbers. The usual topology on $\R$ consists of those sets $U\subseteq\R$ so that, for each $x\in U$, there exists $\varepsilon>0$ so that the interval
\[
(x-\varepsilon,x+\varepsilon)=\{y\in\R\mid x-\varepsilon<y<x+\varepsilon\}
\]
is entirely contained in $U$. It is not hard to see that a function $f\colon\R\to\R$ is continuous with respect to this topology if and only if it satisfies the usual epsilon-delta definition of continuity (see Exercise~\ref{exer:real}).
\end{example}

A continuous map $f\colon X\to Y$ between topological spaces is said to be an \emphind{open mapping} provided $f[U]=\{f(x)\mid x\in U\}$ is open in $Y$ for any open $U\subseteq X$. Similarly,  $f\colon X\to Y$ is said to be a \emphind{closed mapping} provided $f[C]=\{f(x)\mid x\in C\}$ is closed in $Y$ for any closed $C\subseteq X$. Further, $f$ is said to be an \emphind{embedding} provided it is injective and $f^{-1}\colon\im(f)\to X$ is also continuous. Finally, $f$ is a \emphind{homeomorphism} provided it is a bijection and both $f$ and $f^{-1}$ are continuous.

Since open sets are closed under unions, for any subset $S$ of a topological space $(X, \tau)$, there is a largest open set, $\int(S)$, that is contained in $S$. The set $\int(S)$ is called the \emphind{interior} of $S$; in a formula, 
\[\int(S) = \bigcup\{U \subseteq X \ | \ U \subseteq S \text{ and } U \in \tau\}.\] 
We note that $\int$ is the upper adjoint to the inclusion map $\Omega(X) \into \mathcal{P}(X)$, see Exercise~\ref{exer:int-adjoint}. The interior of a set $S$ is sometimes also denoted by $S^\circ$.
Symmetrically, since closed sets are closed under arbitrary intersections, any subset $S$ has a \emphind{closure}, $\cl(S)$, which is defined as the smallest closed set containing $S$; in a formula, 
\[\cl(S) = \bigcap \{C\subseteq X \ | \ S \subseteq C\text{ and } C\text{ is } \tau\text{-closed}\}.\] 
The map $\cl$ is the lower adjoint to the inclusion map $\cC(X) \into \mathcal{P}(X)$. The closure of a set $S$ is sometimes also denoted by $\overline{S}$.

A \emphind{subspace} of a topological space $(X,\tau)$ is given by a subset $Y\subseteq X$ and is equipped with the topology
\[
\tau\upharpoonright Y=\{U\cap Y\mid U\in\tau\}.
\]

Notice that arbitrary intersections of topologies on a fixed set $X$ are again topologies. Accordingly, for any collection $\cS$ of subsets of $X$, there is a least topology containing $\cS$. We call this the \emphind{topology generated by $\cS$}. We denote it by ${\langle}\cS{\rangle}$ and refer to $\cS$ as a \emphind{subbasis} for that topology.

A  \emphind{basis} for a topology $\tau$ on $X$ is a subcollection $\cB\subseteq\tau$ so that for each $U\in\tau$ and $x \in U$, there exists $V\in\cB$ with $x\in V\subseteq U$. The topology generated by a basis $\cB$ is obtained simply by closing it under arbitrary unions (see Exercise~\ref{exer:basis}).

Let $(X_i)_{i \in I}$ be a collection of topological spaces indexed by a set $I$. The \emphind{product space} $\prod_{i\in I} X_i$ is the Cartesian product of the $X_i$ equipped with the topology generated by the subbasis consisting of the sets
\[
\pi_i^{-1}(V), \text{ where }i\in I \text{ and } V\subseteq X_i \text{ is open in  }X_i.
\]

Let $X$ be a topological space. We recall the four \emphind{separation axioms} that may hold for $X$.
\begin{itemize}
\item $X$ is $T_0$ (or Kolmogorov) provided, for all $x,y\in X$ with $x\neq y$ there is an open $U\subseteq X$ which contains exactly one of $x$ and $y$;
\item $X$ is $T_1$ (or Fr\'echet) provided, for all $x,y\in X$ with $x\neq y$ there is an open $U\subseteq X$ with
$x\in U$ and $y\not\in U$;
\item $X$ is $T_2$ (or Hausdorff) provided, for all $x,y\in X$ with $x\neq y$ there are opens $U,V\subseteq X$ with
$x\in U$ and $y\in V$ and $U\cap V=\emptyset$;
\item $X$ is $T_3$ (or regular) provided $X$ is $T_1$ and, for all $x\in X$ and $C\subseteq X$ closed with $x\not\in C$ there are opens $U,V\subseteq X$ with $x\in U$ and $C\subseteq V$ and $U\cap V=\emptyset$;
\item $X$ is $T_4$ (or normal) provided $X$ is $T_1$ and, for all $C,D\subseteq X$ closed and disjoint there are opens $U,V\subseteq X$ with $C\subseteq U$ and $D\subseteq V$ and $U\cap V=\emptyset$.
\end{itemize}

Let $S\subseteq X$ where $X$ is a topological space. An \emphind{open cover} $\cC$ of $S$ is a collection of open sets $\cC\subseteq\Omega X$ so that $S\subseteq \bigcup\cC$. A subset $K\subseteq X$ is \emphind{compact} provided every open cover $\cC$ of $K$ contains a finite subcover, that is, a finite $\cC'\subseteq\cC$ which is also a cover of $K$. We now recall a result, which requires a non-constructive principle, and is often useful for proving compactness.

\begin{alexanderthm}\label{lem:alexander}
Let $X$ be a topological space and $\cS$ a subbasis for the topology on $X$. If every cover $\cC\subseteq\cS$ of $X$ has a finite subcover, then $X$ is compact.
\end{alexanderthm}

Any proof of the Alexander Subbase Theorem must use a non\hyp{}constructive principle, such as Zorn's Lemma. We will not enter into many set-theoretic considerations in this book, but we do cite the statement of Zorn's Lemma, which will also be crucially used in Chapter~\ref{ch:priestley} in the proof of the Stone representation theorem (Theorem~\ref{thm:DLrep}).\endnote{For more information about equivalences between various choice principles, see for example the books~\cite{Jec1973, How1998}.}

\begin{zornslemma}\index{Zorn's Lemma}\index{Axiom of Choice}\label{lem:zorn}
Let $S$ be a non-empty partially ordered set such that if $C \subseteq S$ is totally ordered, then there exists an upper bound $c$ of $C$ in $S$. Then $S$ has a maximal element, i.e., there exists $s \in S$ such that for any $s' \in S$, if $s' \geq s$, then $s' = s$.
\end{zornslemma}

\begin{proof}[Proof that Zorn's Lemma implies Alexander Subbase Theorem.] We ask the reader to fill in the gaps of this proof sketch in Exercise~\ref{exer:alexander}. We prove the contrapositive statement. Suppose that $X$ is not compact. Zorn's Lemma guarantees that there exists a \emph{maximal} open cover $\cC$ of $X$ which does not have a finite subcover. The subcollection $\cC \cap \cS$ of $\cC$ can be shown to still be a cover of $X$, using the maximality of $\cC$. Now $\cC \cap \cS$ is a cover by elements from the subbasis, which can not have a finite subcover.
\end{proof}

\begin{example}\label{exp:compact}
The subspace $[0,1]=\{x\in \R\mid 0\leq x\leq 1\}$ of $\R$ with the usual topology is compact (see Exercise~\ref{exer:Icompact}).
\end{example}

A topological space $X$ is \emphind{locally compact} provided that, for each $x\in X$ and each $U\in\Omega X$ with $x\in U$, there are $V\in\Omega X$ and $K\subseteq X$ compact so that
\[
x\in V\subseteq K\subseteq U.
\]
Note that, if $X$ is Hausdorff, then compactness implies local compactness, while this is not the case in general (see Exercise~\ref{exer:loccompact}).

\begin{proposition}\label{prop:proj-along-comp}
Let $X$ and $Y$ be topological spaces and $\pi_Y\colon X\times Y \to Y$ the projection onto the second coordinate. If $X$ is compact, then $\pi_Y$ is a closed mapping.
\end{proposition}

\begin{proof}
Let $C\subseteq X\times Y$ be closed and suppose $y\not\in\pi_Y[C]$. That is, for each $x\in X$, we have $(x,y)\not\in C$. Thus, as $C$ is closed,  for each $x\in X$, there are $U_x$ open in $X$ and $V_x$ open in $Y$ with $(x,y)\in U_x\times V_x$ and
\[
C\cap (U_x\times V_x)=\emptyset.
\]
Since $x\in U_x$ for each $x\in X$, the collection $\{U_x\mid x\in X\}$ is an open cover of $X$. Also, as $X$ is compact, there is a finite subset $M\subseteq X$ so that $\{U_x\mid x\in M\}$ covers $X$. Now
letting $V=\bigcap\{V_x\mid x\in M\}$ we have $y\in V$, $V\subseteq Y$ open, and $V\cap\pi_Y[C]=\emptyset$.
\end{proof}

\exercises
\begin{exercise}\label{exer:real}
Consider $\R$ equipped with the collection of subsets $U\subseteq\R$ with the property that, for each $x\in U$, there exists $\varepsilon>0$ so that the interval $(x-\varepsilon,x+\varepsilon)$ is entirely contained in $U$.
\begin{enumerate}
\item Show that $(\R,\tau)$ is a topological space;
\item Show that the collection of intervals $(r,s)$, where both $r$ and $s$ are rational forms a basis for $\R$.
\item Show that $f\colon\R\to\R$ is continuous if and only if, for every $x\in\R$ and every $\varepsilon>0$, there exists $\delta>0$ so that, for all $y\in\R$ with $|x-y|<\delta$ we have $|f(x)-f(y)|<\varepsilon$.
\end{enumerate}
\end{exercise}

\begin{exercise}\label{exer:open-closed-map}
Give examples of continuous maps which are:
\begin{enumerate}
\item neither open nor closed,
\item open but not closed,
\item closed but not open.
\end{enumerate}
\end{exercise}

\begin{exercise}\label{exer:int-adjoint}
Let $X$ be a topological space. Prove that the interior map $\int \colon \mathcal{P}(X) \to \Omega(X)$ is an upper adjoint to the inclusion map $\iota \colon \Omega(X) \into \mathcal{P}(X)$. State and prove the analogous statement for the closure map $\cl \colon \mathcal{P}(X) \to \cC(X)$.
\end{exercise}

\begin{exercise}\label{exer:emb-homeo-subsp}
Let $X$ and $Y$ be topological spaces, and $f\colon X\to Y$ a continuous injection.
\begin{enumerate}
\item Show that $f$ need not be an embedding;
\item Show that $f$ is an embedding if and only if $f$ co-restricted to $\im(f)$ is a homeomorphism. Here, the \emphind{co-restriction} of $f\colon X\to Y$ to $S\subseteq Y$, where $\im(f)\subseteq S$, is the function with domain $X$, codomain $S$, and the same action as $f$.
\item Show that if the continuous map $f$ is a bijection, then it is a homeomorphism if and only if it is open and if and only if it is closed.
\item Show that if $X\subseteq Y$ and $f$ is the set inclusion, then $X$ is a subspace of $Y$ if and only if $f$ is an embedding.
\end{enumerate}
\end{exercise}

\begin{exercise}\label{exer:basis}
Let $(X,\tau)$ be a topological space, and $\cB$ and $\cS$ be, respectively, a basis and a subbasis for $\tau$.
\begin{enumerate}
\item  Show that $\cB$ is a subbasis for $\tau$.
\item Show that ${\langle}\cB{\rangle}=\{\bigcup\cC\mid \cC\subseteq\cB\}$.
\item \label{itm:basis-from-subbasis} Let $\cT$ be the closure of $\cS$ under finite intersections. Show that $\cT$ is a basis for $\tau$.
\item Give an example of a basis for a topology which is not closed under binary intersections.
\item Let $Y$ be a topological space. Show that $f\colon Y\to X$ is continuous if and only if $f^{-1}(U)$ is open in $Y$ for each $U\in\cS$.
\end{enumerate}
\end{exercise}

\begin{exercise}\label{exer:alexander}
  This exercise (based on \cite[Exercise~3.12.2, p.~221]{Engelking89}) asks you to fill in the details of the proof of the Alexander Subbase Theorem. 
  Let $X$ be a topological space and $\cS$ a subbasis for the topology on $X$.
  \begin{enumerate}
  \item Prove that, in the following sub-poset of $(\mathcal{P}(\Omega(X)), \subseteq)$, 
    \[\mathbf{C} := \{ \cC \subseteq \Omega(X) \ : \ \cC \text{ is a cover of } X \text{ and } \cC \text{ has no finite subcover}\},\] any totally ordered subset $(\cC_i)_{i \in I}$ of $\mathbf{C}$ has an upper bound in $\mathbf{C}$. \hint{Show that $\bigcup_{i \in I} \cC_i$ is in $\mathbf C$..}

    By Zorn's Lemma, if $X$ is not compact, pick a maximal element $\cC$ of the poset $\mathbf{C}$.
    \item Prove that, for any open sets $U, V$, if $V \in \cC$ and $U \subseteq V$, then $U \in \cC$.

    \item Prove that, for any finite number of open sets $U_1, \dots U_n$, if $U_i \not\in \cC$ for every $1 \leq i \leq n$, then $\bigcap_{i=1}^n U_i \not\in \cC$. \hint{Use the maximality of $\cC$ to get finite subcovers $\mathcal{F}_i$ of $\cC \cup \{U_i\}$ for every $i$, and show that $\mathcal{F} := \{\bigcap_{i=1}^n F_i : (F_i)_{i=1}^n \in \mathcal{F}_1 \times \cdots \times \mathcal{F}_n\}$ is then a finite subcover of $\cC \cup \{\bigcap_{i=1}^n U_i\}$.}
      \item Conclude that $\cC \cap \cS$ is a cover of $X$ that does not have a finite subcover.
  \end{enumerate}
\end{exercise}


\begin{exercise}\label{exer:sep}
Show that $T_4$ implies $T_3$, which implies $T_2$, which implies $T_1$, which implies $T_0$. Further, show that all these implications are strict.
\end{exercise}

\begin{exercise}\label{exer:Hausdorff}
Show that a topological space $X$ is Hausdorff if and only if the diagonal
\[
\Delta_X=\{(x,x)\mid x\in X\}
\]
is closed in the product topology on $X\times X$.
\end{exercise}


\begin{exercise}\label{exer:comp}
Suppose $X$ is a compact topological space.
\begin{enumerate}
\item Let $C\subseteq X$ be closed. Show that $C$ is compact.
\item Find a compact space $X$ with a compact subset $K$ which is not closed.
\item Show that if $X$ is Hausdorff and $K\subseteq X$ is compact, then $K$ is closed.
\end{enumerate}
\end{exercise}

\begin{exercise}\label{exer:Icompact}
Show that the subspace $[0,1]=\{x\in \R\mid 0\leq x\leq 1\}$ of $\R$ with the usual topology is compact.
\end{exercise}

\begin{exercise}\label{exer:compHaus-normal}
Show that any compact Hausdorff space is normal. \hint{Show first that it is regular.}
\end{exercise}

\begin{exercise}\label{exer:Tychonoff}
(Tychonoff's Theorem) Look up a proof (e.g. \cite[Theorem 3.2.4, p.138]{Engelking89}) or show that the product of compact spaces is again compact.
\end{exercise}


\begin{exercise}\label{exer:loccompact}
\begin{enumerate}
	\item Show that any compact Hausdorff space is locally compact. \hint{Use that compact Hausdorff spaces are regular, as proved in Exercise~\ref{exer:compHaus-normal}.}
	\item Find a topological space which is compact but not locally compact.
\end{enumerate}
\end{exercise}

\begin{exercise}[Quotient space]\label{exer:quotspace}
Let $X$ be a topological space and $\equiv$ an equivalence relation on $X$. The \emphind{quotient space} of $X$ by $\equiv$ is the space based on $X/\equiv$ whose open sets are those $U\subseteq X/\equiv$ such that
\[
q^{-1}(U)=\bigcup\{[x]_\equiv\mid [x]_\equiv\in U\}=\{x\in X\mid [x]_\equiv\in U\}
\]
is open in $X$, where $q\colon X\to X/\equiv, x\mapsto [x]_\equiv$ is the canonical quotient map.
\begin{enumerate}
	\item Show that the topology on $X/\equiv$ is the finest topology on $X/\equiv$ making $q\colon X\to X/\equiv$ continuous.
	\item Show that $X/\equiv$ is a $T_1$ space if and only if every equivalence class of $\equiv$ is closed in $X$.
	\item Show that if $X/\equiv$ is a Hausdorff space, then $\equiv$ is necessarily a closed as a subset of the product space $X\times X$. 	
\item Shoe that if the quotient map is open, then $X/\equiv$ is a Hausdorff if and only if $\equiv$ is a closed in $X\times X$. 	
	\item Show that if $f\colon X\twoheadrightarrow Y$ is a continuous surjection, then $f$ factors through the canonical quotient map $q\colon X\to X/\ker(f)$ by a unique continuous bijection $\tilde{f}\colon X/\ker(f)\to Y$.
	\item Give an example in which $\tilde{f}$ is not a homeomorphism.
	\item Show that if $f$ is open or closed then $\tilde{f}$ is a homeomorphism. But show by giving an example that this condition is not necessary.
\end{enumerate}
\end{exercise}

\section{Topology and order}\label{sec:TopOrd}

Let $X$ be a topological space. The \emphind{specialization order} on $X$ is given by
\[
x\leq y \iff \forall \ U\in\Omega X\ (\,x\in U\implies y\in U\,).
\]
This relation is clearly reflexive and transitive and thus, for any topological space, the specialization order is a preorder on $X$. It is not hard to see that it is a partial order if and only if $X$ is $T_0$. Moreover, for any $y \in X$, the principal down-set, ${\downarrow}y$, of $y$ in the specialization order is the closure of the singleton set $\{y\}$. In particular, $T_1$ spaces can be characterized as those spaces having a trivial specialization order. The reader is asked to prove these statements in Exercise~\ref{exer:specorder}.

A subset of a topological space $X$ is said to be \emphind{saturated} provided it is an up-set in the specialization order. Note that a subset $K\subseteq X$ is compact if and only if its saturation ${\uparrow} K$ is compact. We denote by $\KS X$ the collection of \emphind{compact-saturated} subsets of $X$ (i.e., subsets of $X$ that are both compact and saturated). As we will see later on, beyond the Hausdorff setting, but in the presence of compactness, $\KS X$ is in many aspects the right generalization of the closed subsets. The following fact is often useful and illustrates the consequence of compactness in terms of the specialization order.

\begin{proposition}\label{prop:compact-implies minpoints}
Let $X$ be a $T_0$ and $K\subseteq X$ compact, then $K\subseteq {\uparrow}\min(K)$. In particular, if $K$ compact saturated then $K= {\uparrow}\min(K)$. 
\end{proposition}

\begin{proof}
Let $D$ be a down-directed set in $X$ equipped with its specialization order. Then $\{({\downarrow }x)^c\mid x\in D\}$ is an up-directed collection of open subsets and if $K$ contains no lower bounds of $D$, then $\{({\downarrow }x)^c\mid x\in D\}$ is an open cover of $K$. In this case, by compactness, it follows that there is $x\in D$ with $K\subseteq ({\downarrow }x)^c$. In particular, $x\not\in K$ and thus $D\not\subseteq K$. By contraposition we have proved that if $D\subseteq K$ is down-directed, then there is a lower bound of $D$ in $K$. 

Now let $x\in K$ and, by the Hausdorff Maximality Principle, let $C$ be a maximal chain in $K$ containing $x$. Then, by the above argument, $C$ has a lower bound $x'\in K$. Now by maximality of $C$, it follows that $x'\in\min(K)$ and thus $K\subseteq {\uparrow}\min(K)$.
\end{proof}

\subsection*{The lattice of all topologies on a set}

Let $X$ be a set. Note that the collection  
\[
\Top(X):=\{\tau\in\cP(\cP(X))\mid \tau \text{ is a topology}\}.
\]
is closed under arbitrary intersections and thus (see Exercise~\ref{exe:complattsuff} in Chapter~\ref{ch:order}) it is a complete lattice in the inclusion order. Infima are given by intersections, while suprema are given by the topologies generated by unions. The least topology on $X$ is the indiscrete topology, while the largest is the discrete topology. We will often make use of the binary join of topologies on a given set $X$.

The interaction of compactness and the Hausdorff separation axiom is illuminated by looking at $\Top(X)$ as a complete lattice. Indeed, by inspecting the definitions (see Exercise~\ref{ex:down-and-up-in-Top}), note that
\[
T_2(X):=\{\tau\in\Top(X)\mid \tau \text{ is } T_2\}
\]
is an up-set in $\Top(X)$, while
\[
T_{\it Comp}(X):=\{\tau\in\Top(X)\mid (X,\tau) \text{ is compact}\}
\]
is a down-set in $\Top(X)$. It follows that the set of compact-Hausdorff topologies on a set $X$ form a convex\index{convex} subset of $\Top(X)$. The following very useful result tells us that it is in fact an anti-chain.

\begin{proposition}\label{prop:compHaus-incomp}
Let $X$ be a set and $\sigma$ and $\tau$ topologies on $X$ with $\sigma\subseteq\tau$. If $\sigma$ is Hausdorff and $\tau$ is compact, then $\sigma=\tau$.
\end{proposition}

\begin{proof}
Since $T_2(X)$ is an up-set and $\Top_{\it Comp}(X)$ is a down-set, the hypotheses on $\sigma$ and $\tau$ imply that both are simultaneously compact and Hausdorff. Now we have the following string of (bi)implications for any subset $U\subseteq X$:
\begin{align*}
U\in\tau  &\iff U^c\in \cC(X,\tau)\\
               & \iff U^c\in \KS(X,\tau) \\
               & \implies U^c\in \KS(X,\sigma)\\
               & \iff U^c\in \cC(X,\sigma)\iff U\in\sigma.
\end{align*}
Here the second and third bi-implications hold because in compact-Hausdorff spaces being closed is equivalent to being compact-saturated (see Exercise~\ref{exer:compHaus}), and the implication in between holds as any set which is compact in a bigger topology remains so in the smaller topology, while saturation is vacuous in $T_1$ spaces.
\end{proof}

\subsection*{Order-topologies}
From topology we get order, but it is also possible to go the other way. Especially in computer science applications where second-order structure such as a topology is difficult to motivate, topologies induced by orders play an important r\^ ole; see also the applications to domain theory in Chapter~\ref{chap:DomThry}.

Let $(P,\leq)$ be an ordered set.
\begin{itemize}
\item As we have seen, for any topological space, the closures of points are equal to their principal down-sets for the specialization order. We may proceed in the converse direction: The \emphind{upper topology} on $P$ is the least topology in which ${\downarrow} p$ is closed for every $p\in P$. That is, the upper topology is given by
\[
\iota^\uparrow(P)=\langle({\downarrow} p)^c\mid p\in P\rangle.
\]
\item The \emphind{lower topology} on $P$ is defined order-dually. That is,
\[
\iota^\downarrow(P)=\langle({\uparrow} p)^c\mid p\in P\rangle.
\]
\item The \emphind{interval topology} on $P$ is the join of the upper and lower topologies. That is,
\[
\iota(P)=\iota^\uparrow(P)\vee\iota^\downarrow(P).
\]
The usual topology on the reals is in fact the interval topology given by the usual order on the reals.
\item The  \emphind{Scott topology} on $P$ consists of those up-sets which are inaccessible by directed suprema. That is, an up-set $U\subseteq P$ is Scott open if and only if $\bigvee D\in U$ implies $U\cap D\neq\emptyset$ for all directed subsets $D\subseteq P$. We denote the Scott topology on $P$ by $\sigma(P)$.

Of course one can also consider the order-dual of the Scott topology but this is not so common as the motivation for having closed sets which are stable under directed joins comes from a model of computing in which a computation is considered as the directed join of all its partial computations or finite approximations.

\item In terms of two-sided topologies involving the Scott topology, the following topology is more useful than the `double-Scott' topology (obtained by joining the Scott topology and its dual). The \emphind{Lawson topology} on $P$ is the join of the Scott and the lower topologies. That is,
\[
\lambda(P)=\sigma(P)\vee\iota^{\downarrow}(P).
\]

\item The \emphind{Alexandrov topology} on $P$ is the largest topology on $P$ yielding $\leq$ as its specialization order. That is,
\[
\alpha(P)=\{ U\subseteq P\mid U\text{ is an up-set}\}.
\]
Note that the join of the Alexandrov and dual Alexandrov topologies is the discrete topology on $P$.
\end{itemize}
The upper, the Scott, and the Alexandrov topologies all have the original order $\leq$ as their specialization order. In fact, if we denote by $\Top(P,\leq)$ the lattice of topologies on $X$ yielding $\leq$ as their specialization order, it is not hard to see that this is precisely the closed interval $[\iota^\uparrow(P),\alpha(P)]$ in $\Top(P)$.

Clearly the order-duals of these topologies have the reverse of $\leq$ as their specialization order and the two-sided topologies such as the interval topology, the double Scott topology, the Lawson topology, and the discrete topology are all $T_1$ and thus have trivial specialization order.

\exercises
\begin{exercise}
  \label{ex:down-and-up-in-Top}
  Prove that $T_2(X)$ is an up-set in $\Top(X)$, and that $T_{Comp}(X)$ is a down-set in $\Top(X)$.
\end{exercise}
\begin{exercise}\label{exer:closedmap}
Let $X$ be a compact space, $Y$ a Hausdorff space, and $f\colon X\to Y$ be a continuous map.
\begin{enumerate}
\item Show that $f$ is closed. \hint{Use Proposition~\ref{prop:compHaus-incomp}.}
\item Show that if $f$ is a bijection, then it is a homeomorphism.
\item Show that  the  co-restriction of $f$ to its image $X\twoheadrightarrow\Im(f), x\mapsto f(x)$ is  a quotient map. In particular, as soon as $f$ is surjective, it is a quotient maps, cf. Exercise~\ref{exer:quotspace}.
\end{enumerate}
\end{exercise}

\begin{exercise}\label{exer:specorder}
Let $X$ be a topological space.
\begin{enumerate}
\item Show that the specialization order on $X$ is a preorder;
\item Show that $X$ is $T_0$ if and only if the specialization order on $X$ is an order;
\item Show that $X$ is $T_1$ if and only if the specialization order on $X$ is trivial. That is, $x\leq y$ if and only if $x=y$;
\item Show that $x\leq y$ in the specialization order if and only if $x\in\overline{\{y\}}$. That is, $\overline{\{y\}}={\downarrow}y$;
\item Show that a subset $S\subseteq X$ is an intersection of open sets if and only if it is an up-set in the specialization order.
\end{enumerate}
\end{exercise}

\begin{exercise}\label{exer:interval-ordertop}
Let $X$ be a set and $\leq$ an order on $X$. Show that a topology $\tau$ on $X$ has $\leq$ as its specialization order if and only if
\[
\iota^\uparrow(X,\leq)\subseteq\tau\subseteq\alpha(X,\leq).
\]
\end{exercise}

\begin{exercise}\label{exer:cont-implies-op}
Show that if a function $f\colon X\to Y$ between topological spaces is continuous then it is order preserving with respect to the specialization orders on $X$ and $Y$. Give an example to show that the converse is false.
\end{exercise}

\begin{exercise}\label{exer:TOPfin}
Show that if $P$ is a finite ordered set then $\iota^\uparrow(P)=\alpha(P)$. Further show that, for any two ordered sets $P$ and $Q$, a map $F\colon P\to Q$ is order preserving if and only if it is continuous with respect to the Alexandrov topologies on $P$ and $Q$. 

{\it Remark.} Using terminology that we will introduce in Chapter~\ref{ch:categories}, Definition~\ref{dfn:iso-between-cats}, this exercise shows that
the category of finite ordered sets is isomorphic to the category of finite $T_0$ topological spaces, for further details see Example~\ref{exa:topfiniso} in that chapter.
\end{exercise}

\begin{exercise}\label{exer:compHaus}
Let $X$ be a compact Hausdorff space and $K\subseteq X$. Show that $K$ is closed if and only if it is compact if and only if it is compact saturated.
\end{exercise}

\section{Compact ordered spaces}\label{sec:comp-ord-sp}
In this section we show that there is an isomorphism between certain compact spaces equipped with an order, first introduced by Nachbin, and certain $T_0$ spaces known as stably compact spaces. These spaces provide a well-behaved generalization of compact Hausdorff spaces and contain the spaces dual to distributive lattices, which are the main object of study of this book.

\begin{definition}\label{dfn:comp-ord-sp}
An \emphind{ordered space} is a triple $(X,\tau,\leq)$ such that
\begin{itemize}
\item $(X,\tau)$ is a topological space;
\item $(X,\leq)$ is an ordered set;
\item $\leq\ \subseteq X\times X$ is closed in the product topology.
\end{itemize}
An ordered space is said to be a \emphind{compact ordered space} provided the underlying topological space is compact.
\end{definition}

A \emph{morphism}\index{ordered space!morphism}\index{morphism between ordered spaces} from an ordered space $(X, \tau_X, \leq_X)$ to an ordered space $(Y, \tau_Y, \leq_Y)$ is a function $f \colon X \to Y$ that is both continuous as a map from the space $(X, \tau_X)$ to $(Y,\tau_Y)$ and order-preserving as a map from the poset $(X, \leq_X)$ to $(Y, \leq_X)$. An \emphind{order-homeomorphism} between ordered spaces is a morphism that is both a homeomorphism and an order-isomorphism. For an equivalent definition of order-homeomorphism, see Exercise~\ref{exer:orderhomeo}.

\begin{proposition}
Let $X$ be an ordered space. Then the underlying topological space is Hausdorff.
\end{proposition}

\begin{proof}
Since $X$ is an ordered space, $\leq$ closed in $X\times X$ equipped with the product topology. Thus $\geq$ is also closed in $X\times X$ with the product topology and it follows that $\Delta_X=\,\leq\cap\geq$ is closed in $X\times X$ with the product topology. But this is equivalent to $X$ being Hausdorff, see Exercise~\ref{exer:Hausdorff}.
\end{proof}

The following proposition is an important technical tool in the study of compact ordered spaces.

\begin{proposition}\label{prop:cos-closed}
Let $X$ be a compact ordered space and $C\subseteq X$ a closed subset of $X$. Then ${\uparrow}C$ and ${\downarrow}C$ are also closed. In particular, ${\uparrow}x$ and ${\downarrow}x$ are closed for all $x\in X$.
\end{proposition}

\begin{proof}
If $C\subseteq X$ is closed in $X$, then $C\times X$ is closed in $X\times X$ equipped with the product topology. Now, as $X$ is an ordered space it follows that $\leq$ is closed and thus $(C\times X)\,\cap\leq$ is closed in $X\times X$. By Proposition~\ref{prop:proj-along-comp} it follows that
\[
\pi_2[(C\times X)\,\cap\leq]={\uparrow}C,
\]
where $\pi_2\colon X\times X\to X$ is the projection on the second coordinate, is closed in $X$. Projecting $(C\times X)\,\cap\geq$ on the second coordinate shows that ${\downarrow}C$ is closed. Finally, as $X$ is Hausdorff, it is in particular $T_1$ and thus the singletons $x$ are all closed. It follows that ${\uparrow}x$ and ${\downarrow}x$ are closed for all $x\in X$.
\end{proof}

We can now derive the following very useful \emphind{order-separation property} for compact ordered spaces.

\begin{proposition}\label{prop:cos-ordsep}
Let $X$ be a compact ordered space. For all $x,y\in X$, if $x\nleq y$, then there are disjoint sets $U,V\subseteq X$ with $U$ an open up-set containing $x$ and $V$ an open down-set containing $y$.
\end{proposition}

\begin{proof}
Let $x,y\in X$ with $x\nleq y$. Then ${\uparrow}x$ and ${\downarrow} y$ are disjoint. Also, by Proposition~\ref{prop:cos-closed}, the sets ${\uparrow}x$ and ${\downarrow} y$ are closed. Now, since $X$ is compact ordered, it is compact Hausdorff and therefore also normal (see Exercise~\ref{exer:compHaus-normal}).  Thus there are open disjoint sets $U,V\subseteq X$ with ${\uparrow}x\subseteq U$ and ${\downarrow} y\subseteq V$. Now let $U'=({\downarrow}U^c)^c$ and $V'=({\uparrow}V^c)^c$, then one may verify that $U'$ is an open up-set, $V'$ is an open down-set, and we have
\[
x\in U' \ \text{ and }\ y\in V' \ \text{ and }\ U'\cap V'=\emptyset.\qedhere
\]
\end{proof}
To any compact ordered space, we now associate two $T_0$ spaces. These spaces will usually not be $T_1$, as long as the order on $X$ is non-trivial. To be specific, if $(X,\tau,\leq)$ is a compact ordered space, then we define
\[
\tau^{\uparrow}=\tau\cap \Up(X,\leq)
\]
and
\[
\tau^{\downarrow}=\tau\cap \Down(X,\leq).
\]
In other words, $\tau^{\uparrow}$ is the intersection of the topology $\tau$ and the Alexandrov topology on $(X,\leq)$ and $\tau^{\downarrow}$ is the intersection of the topology $\tau$ and the dual Alexandrov topology on $(X,\leq)$. Accordingly, $\tau^{\uparrow}$  and $\tau^{\downarrow}$ are indeed topologies on $X$. We will often denote the topological space underlying the original ordered space $(X,\tau, \leq)$ simply by $X$, the space $(X,\tau^{\uparrow})$ by $X^{\uparrow}$, and the space $(X,\tau^{\downarrow})$ by $X^{\downarrow}$.

Note that if $(X,\tau,\leq)$ is a compact ordered space, then so is $(X,\tau,\geq)$. Thus any property of the spaces $X^{\uparrow}$ and their relation to $\leq$ implies that the order-dual property is true for the spaces $X^{\downarrow}$ and we will not always state both.

\begin{proposition}\label{prop:compord-specord}
Let $(X,\tau,\leq)$ be a compact ordered space. Then the specialization order of $X^{\uparrow}$ is $\leq$ and in particular $X^{\uparrow}$ is a $T_0$ space.
\end{proposition}

 \begin{proof}
 For each $x\in X$, ${\downarrow}x$ is closed in $(X,\tau)$ and it is a down-set. Thus ${\downarrow}x$ is closed in in $X^{\uparrow}$. Thus $\iota^{\uparrow}(X,\leq)\subseteq \tau^{\uparrow}$. Also, clearly
 $\tau^{\uparrow}\subseteq \Up(X,\leq)=\alpha(X,\leq)$ and thus the specialization order of $X^{\uparrow}$ is $\leq$ (see Exercise~\ref{exer:interval-ordertop}).
 \end{proof}

 A crucial fact, given in the following proposition, which will enable us to come back to a compact ordered space $X$ from $X^{\uparrow}$, is the fact that $X^{\uparrow}$ and $X^{\downarrow}$ are inter-definable by purely topological means without using the data of the original compact ordered space.

\begin{proposition}\label{prop:Fdown=Kup}
Let $(X,\tau,\leq)$ be a compact ordered space. Then
\[
V\in\tau^{\downarrow} \quad\iff\quad V^c\in\KS(X^{\uparrow}).
\]
\end{proposition}

 \begin{proof}
Note that $\cC(X^{\downarrow})=\cC(X,\tau)\cap\,\Up(X,\leq)$. Also, by definition, $\KS(X^{\uparrow})$ consists of those subsets of $X$ that are both compact with respect to $\tau^{\uparrow}$ and belong to $\Up(X,\leq)$. Thus we need to show that if $S\in\Up(X,\leq)$ then $S$ is closed relative to $\tau$ if and only if it is compact relative to $\tau^{\uparrow}$.

Let $S\in\Up(X,\leq)$. If $S$ is  closed relative to $\tau$, then $S$ is compact relative to $\tau$ (see Exercise~\ref{exer:comp}). But then it is also compact relative to the smaller topology  $\tau^{\uparrow}$ as required. For the converse, suppose now that $S$ is compact relative to $\tau^{\uparrow}$ and let $y\not\in S$. For each $x\in S$, since $x\nleq y$, by Proposition~\ref{prop:cos-ordsep}, there are disjoint sets $U_x,V_x\subseteq X$ with $U_x$ an open up-set containing $x$ and $V_x$ an open down-set containing $y$. It follows that the $(U_x)_{x\in S}$ is an open cover of $S$ relative to $\tau^{\uparrow}$. Thus by compactness, there is a finite subset $F\subseteq S$ so that $(U_x)_{x\in F}$ covers $S$. Let
\[
V=\bigcap\{V_x\mid x\in F\}
\]
then $V$ is disjoint from the union of the $(U_x)_{x\in F}$ and thus from $S$. Also, $V$ is open relative to $\tau^{\downarrow}$ and $y\in V$. That is, we have shown that $S$ is closed relative to $\tau^{\downarrow}$.
 \end{proof}



We are now ready to introduce a class of (unordered) topological spaces called \emph{stably compact spaces}. Stably compact spaces have a fairly complex definition but, as we will see, they are in fact none other than those spaces which occur as $X^{\uparrow}$ for $X$ a compact ordered space.

Before we give the definition (Definition~\ref{dfn:stab-comp-sp}), we need to identify two more properties of spaces, both related to the interaction of compactness and intersection. First, we call a compact space \emphind{coherent} provided the intersection of any two compact-saturated subsets is again compact. A space is called \emphind{well-filtered} provided for any filtering collection $\cF$ of compact-saturated sets and any open $U$ we have
 \[\bigcap\cF\subseteq U \quad \implies \quad \exists K\in\cF\quad K\subseteq U.\]
One can show that if $X$ is well-filtered, then the collection of compact-saturated subsets of $X$ is closed under filtering intersections (see Exercise~\ref{exer:well-filtered}).  Notice that if a space is both coherent and well-filtered then the collection of compact saturated sets is actually closed under arbitrary intersections. Also notice that compact Hausdorff spaces have both these properties since the compact-saturated sets are just the closed sets.

\begin{definition}\label{dfn:stab-comp-sp}
 A \emphind{stably compact space} is a topological space which is $T_0$, compact, locally compact, coherent, and well-filtered.
\end{definition}
 
 \begin{example}\label{ex:Sierpinski}
 Note that any finite $T_0$ space is stably compact. A particularly important stably compact space is the \emphind{Sierpinski space}
 \[
 \utwo^\uparrow=(\{0,1\},\{\emptyset,\{1\},\{0,1\}\}).
 \]
 \end{example}
 
 Using Proposition~\ref{prop:Fdown=Kup} it is not hard to see that $X^\uparrow$ is stably compact whenever $X$ is a compact ordered space. We will now show that this is in fact one direction of a one-to-one correspondence between compact ordered spaces and stably compact spaces. To this end we need the notion of the co-compact dual of a topology.
 
 Let $\tau$ be a topology on a set $X$. The \emphind{co-compact dual} of $\tau$, denoted $\tau^\partial$, is the topology generated by the complements of elements of compact-saturated subsets of $(X,\tau)$. That is,
 \[
 \tau^\partial=\langle K^c\mid K\in\KS(X,\tau)\rangle.
 \]
 The compact-saturated sets are always closed under finite unions, so the collection of their complements is closed under finite intersection and is thus a basis for $\tau^\partial$. In the case of a stably compact space, the compact-saturated sets are also closed under arbitrary intersections (see Exercise~\ref{exer:well-filtered}), so the collection of their complements is already a topology and we have
 \[
 \tau^\partial=\{ K^c\mid K\in\KS(X,\tau)\}.
 \]
 Further we may define the \emphind{patch topology} obtained from $\tau$ to be
 \[
 \tau^p=\tau\vee\tau^\partial.
 \]
 We can now identify how to get back the topology of a compact ordered space $X$ from the topology of $X^\uparrow$.
 
 \begin{proposition}\label{prop:compordaspatch}
 Let $(X,\tau,\leq)$ be a compact ordered space. Then $(\tau^\uparrow)^\partial=\tau^\downarrow$ and
 $(\tau^\uparrow)^p=\tau$.
 \end{proposition}
 
 \begin{proof}
 The first equality is a just a restatement of Proposition~\ref{prop:Fdown=Kup} in terms of the co-compact dual topology and once we have this, we may observe that Proposition~\ref{prop:cos-ordsep} tells us, among other things, that $\tau^\uparrow\vee(\tau^\uparrow)^\partial=\tau^\uparrow\vee\tau^\downarrow$ is a Hausdorff topology which is contained in $\tau$. But, by Proposition~\ref{prop:compHaus-incomp}, if a Hausdorff topology is below a compact topology, then in fact they are equal, so $(\tau^\uparrow)^p=\tau$ as desired.
 \end{proof}
 
 
 
 \begin{theorem}\label{thrm:COSpace-StabCompSp}
 The assignments
 \[
 (X,\tau,\leq)\ \mapsto \ (X,\tau^\uparrow)
 \]
 and
 \[
 \qquad(X,\sigma)\quad \mapsto \ (X,\sigma^p,\leq_\sigma)
 \]
 establish a one-to-one correspondence between compact ordered spaces and stably compact spaces.%
 \endnote{The correspondence between stably compact spaces and compact ordered spaces in the form given here originates with the Compendium on Continuous Lattices \cite{Getc80}. More focused sources presenting the correspondence and the relation with the co-compact dual of a topology are \cite{Jung2004,Lawson2011}.}
 \end{theorem}
 
 
 \begin{proof}
 It is left as Exercise~\ref{exer:Xupstabcomp} to show that if $(X,\tau,\leq)$ is a compact ordered space, then $X^\uparrow$ is stably compact. Here, we show that if $(X,\sigma)$ is a stably compact space then $(X,\sigma^p,\leq_\sigma)$ is compact ordered.
 
 As a first step, we show that $\leq_\sigma$ is closed relative to the product topology induced by the patch topology. Let $x,y\in X$ with $x\nleq_\sigma y$. Then, by definition of the specialization order, there is an open set $U\in\sigma$ with $x\in U$ and $y\not\in U$. By local compactness of $\sigma$ there is $V\in\sigma$ and $K\subseteq X$ which is compact such that $x\in V\subseteq K\subseteq U$. Since ${\uparrow}K$ is also compact and $K\subseteq {\uparrow}K\subseteq U$ since $U$ is an upset in the specialisation order, we may assume, without loss of generality, that $K\in\KS(X,\sigma)$. It follows that $V, K^c\in\sigma^p$, $x\in V$, $y\in K^c$, and, because $V$ is an up-set and $V\cap K^c=\emptyset$, we have
 \[
 (V\times K^c)\;\cap \leq \; =\emptyset.
 \]
 Thus we have shown that an arbitrary element $(x,y)$ of the complement of $\leq_\sigma$ lies in a basic open of the product topology which is disjoint from $\leq_\sigma$ as required.
 
 Next we show that $(X,\sigma^p)$ is compact. By the Alexander Subbase Theorem, it suffices to show that covers by subbasic opens have finite subcovers. We use the subbasis for $\sigma^p$ given by
 \[
 \cS=\{K^c\mid K\in\KS (X,\sigma)\}\cup\sigma.
 \]
 Now let $\cC\subseteq\cS$ be a cover of $X$. Define
 \[
 \cC_\cK=\{K\in\KS (X,\sigma)\mid K^c\in\cC\}
 \]
 and
 \[
 \cC_\sigma=\cC\cap\sigma.
 \]
 Then the fact that $\cC$ covers $X$ implies that
 \[
 \bigcap\cC_\cK\subseteq\bigcup\cC_\sigma.
 \]
 Notice that $U=\bigcup\cC_\sigma$ is open in $(X,\sigma)$ and that 
 \[
 \cC'_\cK=\{\bigcap \cL\mid \cL\subseteq \cC_\cK \text{ is finite} \}
 \]
 is, by coherence, a collection of compact-saturated subsets of $(X,\sigma)$, which is moreover filtering. Thus, by well-filteredness, it follows that there is a finite $ \cL\subseteq \cC_\cK$ with
 \[
 \bigcap\cL\subseteq\bigcup\cC_\sigma.
 \]
 Now, this means that $\cC_\sigma$ is an open cover of the compact-saturated set $\bigcap\cL$. By compactness there is a finite subcover $\cC'_\sigma$ of $\bigcap\cL$. This in turn is equivalent to saying that
 \[
 \cC'=\{K^c\mid K\in \cL\}\cup\cC'_\sigma
 \]
 is a finite subcover of the original cover $\cC$ as required.
 
 Thus it follows that $(X,\sigma^p,\leq_\sigma)$ is indeed a compact ordered space whenever $(X,\sigma)$ is a stably compact space. 
 
 
 Further, the combination of Proposition~\ref{prop:compord-specord} and Proposition~\ref{prop:compordaspatch} implies that the composition of the two assignments gives the identity on compact ordered spaces. It remains to show that the reverse composition yields the identity of stably compact spaces. 
  To this end, let $(X,\tau)$ be a stably compact space. Since $\sigma\subseteq\sigma^p$ and $\sigma\subseteq\Up(X,\leq_\sigma)$, it follows that $\sigma\subseteq\sigma^p\cap\,\Up(X,\leq_\sigma)$. The fact that $\sigma^p\cap\,\Up(X,\leq_\sigma)\subseteq\sigma$ follows from Exercise~\ref{exer:SCSsubbasis}.
 \end{proof}
 
 While the correspondence of the above theorem provides an isomorphism for objects, the natural classes of maps for compact ordered spaces and stably compact spaces are not the same. A natural notion of morphism for stably compact spaces is that of a continuous function. For compact ordered spaces the natural notion of structure preserving map is that of a function which is simultaneously continuous and order preserving.

It is not hard to see that every continuous and order preserving map between compact ordered spaces is continuous for the corresponding stably compact spaces. However, the converse is not true in general. In fact, the continuous and order preserving maps between compact ordered spaces correspond to the so-called proper maps between stably compact spaces (see Exercise~\ref{exer:propermaps}). We will show in Example~\ref{exa:stcomp-kord-iso} of Chapter~\ref{ch:categories} that this correspondence formally yields an isomorphism of categories.

We finish this section by recording a useful `translation' between various properties of subsets of a compact ordered space and its corresponding stably compact space. The proof is left as an instructive exercise in applying Theorem~\ref{thrm:COSpace-StabCompSp}.
\begin{proposition}\label{prop:with-without-order-scs}
Let $(X,\tau,\leq)$ be a compact ordered space, with $(X,\tau^\downarrow)$ and $(X,\tau^\uparrow)$ the stably compact spaces of $\tau$-open up-sets and $\tau$-open down-sets, respectively. For any subset $S$ of $X$,
\begin{enumerate}
  \item $S$ is saturated in $(X,\tau^{\uparrow})$ iff $S$ is an up-set in $(X,\tau,\leq)$ iff the complement of $S$ is saturated in $(X,\tau^{\downarrow})$;
  \item $S$ is closed in $(X,\tau^{\uparrow})$ iff $S$ is a closed down-set in $(X,\tau,\leq)$ iff $S$ is compact and saturated in $(X,\tau^{\downarrow})$;
  \item \label{itm:koisclup} $S$ is compact and open in $(X,\tau^{\uparrow})$ iff $S$ is a clopen up-set in $(X,\tau,\leq)$ iff the complement of $S$ is compact and open in $(X,\tau^{\downarrow})$.
\end{enumerate}
\end{proposition}
 


\exercises
\begin{exercise}\label{exe:compordaltdef}
Let $(X,\tau,\leq)$ be a triple such that $(X,\tau)$ is a topological space and $(X,\leq)$ is a poset. Prove that the following are equivalent:
\begin{enumerate}
\item[(i)] The order $\leq$ is closed in $(X,\tau) \times (X,\tau)$.
\item[(ii)] for every $x, y \in X$, if $x \nleq y$, then there exist open subsets $U$ and $V$ of $(X,\tau)$ such that $x \in U$, $y \in V$, and ${\uparrow}U \cap {\downarrow}V = \emptyset$.
\end{enumerate}
\end{exercise}

\begin{exercise}\label{exer:orderhomeo}
Let $f \colon X \to Y$ be a morphism between ordered spaces. Prove that $f$ is an order-homeomorphism if, and only if, there exists a morphism $g \colon Y \to X$ such that $g \circ f = \id_X$ and $f \circ g = \id_Y$.
\end{exercise}

\begin{exercise}\label{exer:ordnorm}
Ordered variants of regularity and normality also hold for compact ordered spaces. In particular, if $C$ and $D$ are closed subspaces of a compact ordered space with
\[
{\uparrow} C\cap{\downarrow}D=\emptyset
\]
then there are disjoint sets $U,V\subseteq X$ with $U$ an open up-set containing $C$ and $V$ an open down-set containing $D$.
\end{exercise}

\begin{exercise}\label{exer:unitint}
Show that $\mathbb{R}$ with its usual order and topology is an ordered space. Show that the unit interval $[0,1]$ is a compact ordered space.
\end{exercise}

\begin{exercise}\label{exer:prodCOS}
Show that $\btwo=\{0,1\}$ with the discrete topology and the usual order is a compact ordered space. We will denote this space by $\utwo$ to differentiate it from the two-element lattice (or Boolean algebra).

Also show that compact ordered spaces are closed under arbitrary Cartesian product (where the product is equipped with the product topology and the coordinate-wise order). Conclude that for any set $X$, the space $\utwo^X$ is a compact ordered space.
\end{exercise}

\begin{exercise}\label{exer:COSadjunction}
Let $(X,\tau,\leq)$ be a compact ordered space.
\begin{enumerate}
\item Show that  the inclusion $\Up(X,\leq)\hookrightarrow\cP(X)$ has a lower adjoint given by ${\uparrow}(\ )\colon\cP(X)\twoheadrightarrow\Up(X,\leq)$ and an upper adjoint given by $S\mapsto ({\downarrow}S^c)^c$.
\item Conclude that Proposition~\ref{prop:Fdown=Kup} implies that the inclusion $\KS(X^{\uparrow})\hookrightarrow\KS(X)=\cC(X)$ has a lower adjoint given by $C\mapsto{\uparrow}C$ and that the inclusion $\tau^{\uparrow}\hookrightarrow \tau$ has an upper adjoint given by $U\mapsto({\downarrow}U^c)^c$.\endnote{We thank Universit\'e C\^ote d'Azur Master's student J\'er\'emie Marques for suggesting the reformulation of Proposition~\ref{prop:Fdown=Kup} in terms of upper and lower adjoints on opens and compact saturated sets, outlined in this exercise.}
\end{enumerate}
\end{exercise}

\begin{exercise}\label{exer:coherence} \
\begin{enumerate}
\item Show that the finite union of compact sets is compact;
\item Give an example to show that the intersection of two compact sets need not be compact;
\end{enumerate}
\end{exercise}

\begin{exercise}\label{exer:well-filtered}
  \begin{enumerate}
  \item Show that if $X$ is well-filtered then the intersection of any filtering collection of compact-saturated sets is again compact-saturated.
  \item Show that, if a collection $\mathcal{F}$ of subsets of $\mathcal{P}(X)$ is filtering and closed under finite intersections (i.e., for any finite $\mathcal{S} \subseteq \mathcal{F}$, $\bigcap \mathcal{S} \in \mathcal{F}$), then $\mathcal{F}$ is closed under arbitrary intersections.
  \item Conclude that if $X$ is well-filtered, compact, and coherent, then any intersection of compact-saturated sets is again compact-saturated.
  \end{enumerate}
\end{exercise}

\begin{exercise}\label{exer:compHaus-stabcomp} \
 Show that if $X$ is a compact Hausdorff space, then $X$ is stably compact.
\end{exercise}
\begin{exercise}\label{exer:with-without-order}
Deduce Proposition~\ref{prop:with-without-order-scs} from Proposition~\ref{prop:compordaspatch} and Theorem~\ref{thrm:COSpace-StabCompSp}.
\end{exercise}

\begin{exercise}\label{exer:Xupstabcomp} \
 Show that if $X$ is a compact ordered space, then $X^\uparrow$ is stably compact.
\end{exercise}

\begin{exercise}\label{exer:covers} \
Let $X$ be a topological space. Show that all of the following statements are equivalent:
\begin{enumerate}[label=(\roman*)]
\item $\forall\mathcal R\subseteq \Omega X\ (\,\bigcup\mathcal R=X\implies \exists\mathcal R'\subseteq\mathcal R\text{ finite such that }\bigcup\mathcal R'=X)$;\\[-2ex]

\item $\forall\mathcal D\subseteq\Omega X\text{ directed }\ (\,\bigcup\mathcal D=X\implies X\in\mathcal D)$;\\[-2ex]

\item $\forall\mathcal S\subseteq\cC X\ (\,\bigcap\mathcal S=\emptyset\implies \exists\mathcal S'\subseteq\mathcal S\text{ finite such that }\bigcap\mathcal S'=\emptyset)$;\\[-2ex]
\item $\forall\mathcal F\subseteq\cC X\text{ filtering }\ (\,\bigcap\mathcal F=\emptyset\implies \emptyset\in\mathcal F)$;\\[-2ex]

\item $\forall\mathcal R\subseteq\Omega X\ \forall\mathcal S\subseteq\cC X
\ (\,\bigcap\mathcal S\subseteq\bigcup\mathcal R\implies \exists\,\mathcal R'\subseteq\mathcal R
\ \exists\,\mathcal S'\subseteq\mathcal S\text{ both}\\
\text{finite such that }\bigcap\mathcal S'\subseteq\bigcup\mathcal R')$;\\[-2ex]

\item $\forall\mathcal D\subseteq\Omega X\text{ directed }\
\forall\mathcal F\subseteq\cC(X)\text{ filtering }\
(\bigcap\mathcal F\subseteq\bigcup\mathcal D\implies \exists\, F\in\mathcal F\\
\exists\, U\in\mathcal D\text{ such that }F\subseteq U)$.
\end{enumerate}
\end{exercise}

\begin{exercise}\label{exer:}
Let $(X,\tau)$ be a $T_0$ space. Show that $\leq_{\tau^\partial}\ =\ \geq_\tau$.
\end{exercise}

\begin{exercise}\label{exer:SCSsubbasis} \
Let $(X,\tau)$ be a stably compact space, $\cB\subseteq\tau$, and $\cC\subseteq\KS(X,\tau)$ such that
\[
\forall x,y\in X\ (\,x\nleq y\ \implies \ \exists U\in\cB\ \exists\,K\in\cC\ [x\in U\subseteq K, \text{ and }
y\not\in K]\,)
\]
Show that $\mathcal B$ is a subbasis for $\tau$.
\end{exercise}

\begin{exercise}\label{exer:SCSretracts}
Show that continuous retracts of stably compact spaces are again stably compact. Here a \emphind{retract} of a topological space $X$ is a continuous function $f\colon X\to X$ with $f^2=f$.
\end{exercise}

\begin{exercise}\label{exer:propermaps}
Let $f\colon X\to Y$ be a continuous function between topological spaces. We call $f$ a \emphind{proper map} provided the following two properties hold:
\begin{enumerate}
\item  ${\downarrow}f(C)$ is closed whenever $C\subseteq X$ is closed;
\item $f^{-1}[K]$ is compact for any $K\subseteq Y$ which is compact saturated.
\end{enumerate}
Now let $X$ and $Y$ be compact ordered spaces and denote the corresponding pair of stably compact spaces by $X^{\uparrow}, X^{\downarrow}$ and $Y^{\uparrow}, Y^{\downarrow}$, respectively. Further let  $f\colon X\to Y$ be a function between the underlying sets. Show that the following conditions are equivalent:
\begin{enumerate}[label=(\roman*)]
\item $f$, viewed as a map between compact ordered spaces, is continuous and order preserving;
\item $f$, viewed as a map between the stably compact spaces $X^{\uparrow}$ and $Y^{\uparrow}$, is a proper map.
\item $f$, viewed as a map between the stably compact spaces $X^{\downarrow}$ and $Y^{\downarrow}$, is a proper map.
 \end{enumerate}
\end{exercise}

\theendnotes
\setcounter{endnote}{0}

