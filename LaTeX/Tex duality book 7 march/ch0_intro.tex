\chapter*{Introduction}
\addcontentsline{toc}{chapter}{Introduction}

This book is a course in Stone-Priestley duality theory, with applications to logic and theoretical computer science. Our target audience are graduate students and researchers in mathematics and computer science. The main aim of the book is to equip the reader with the theoretical background necessary for reading and understanding current research in duality and its applications. We aim to be didactic rather than exhaustive, while we do give technical details whenever they are necessary to understand what the field is about.

Distributive lattice structures are fundamental to logic, and thus appear throughout mathematics and computer science. The reason for this is that the notion of a distributive lattice is extremely basic: it captures a language containing as its only primitives the logical operators `or', `and', `true' and `false'. Distributive lattices are to the study of logic what rings and vector spaces are to the study of classical algebra.

A mathematical kernel that makes duality theory tick is the fact that the structure of a lattice can be viewed in three equivalent ways. A distributive lattice is all of the following:
\begin{enumerate}
\item a partially ordered set satisfying certain properties regarding upper and lower bounds of finite sets;
\item an algebraic structure with two idempotent monoid operations that interact well with each other;
\item a basis of open sets for a particular kind of order-topological space.
\end{enumerate}
The first part of the book will define precisely the vague notions in this list (`certain properties', `interact well', `a particular kind of'), and will prove that these are indeed three equivalent ways of looking at distributive lattices. The correspondence between algebraic and topological structure in the last two items of the list can be cast in a precise categorical form, and is then called a \emph{dual equivalence} or simply \emph{duality}. A duality identifies an exciting, almost magical, and often highly useful intersection point of algebra and topology.

Historically, Stone showed in the 1930s that distributive lattices are in a duality with spectral spaces: a certain class of topological spaces with a non-trivial specialization order, which are also the Zariski spectra of rings. Stone's duality for distributive lattices is especially well-known in the  more restrictive setting of Boolean algebras, obtained by adding an operator `not' to the lattice signature, which satisfies the usual rules of logic: de Morgan's laws and excluded middle. The restriction of Stone's duality to Boolean algebras shows that they are in a duality with compact Hausdorff zero-dimensional spaces. While the spaces associated to Boolean algebras are more well known than the slightly more general ones associated to distributive lattices, the latter are vastly more versatile, having, among other \emph{all} compact Hausdorff spaces, including connected spaces such as the unit interval of the reals, as retracts.

Nevertheless, Stone's duality for distributive lattices was for at least thirty years seen by many as a lesser sibling of his duality for Boolean algebras, at least partly due to the fact that the spaces that figure are not Hausdorff, and the appropriate functions between the spaces are not all the continuous ones. Priestley's seminal work in the 1970s lifted this obstacle, by giving a first-class role to the specialization order that figures in Stone's spectral spaces. Priestley showed that distributive lattices are in a duality with certain \emph{order}-topological spaces, now called Priestley spaces. The first goalpost in this book is to build up the necessary mathematics to prove Priestley's duality theorem, which we do in Chapter~\ref{ch:priestley}; we also show there how it easily specializes to the case of Boolean algebras. Building up to this first main result, Chapters~\ref{ch:order}~and~\ref{chap:TopOrd} will teach the foundations of order theory and topology, and the \emph{finite} case of duality theory, that we rely on in the rest of the book. 

A unique feature of this book is that, in addition to developing general duality theory for distributive lattices, we also show how it applies in a number of areas within the foundations of computer science, namely, modal and intuitionistic logics, domain theory and automata theory. The use of duality theory in these areas brings to the forefront how much their underlying mathematical theories have in common. It also prompts us to upgrade our treatment of duality theory with various enhancements that are now commonly used in the state-of-the-art research in the field.
Most of these enhancements make use of \emph{operators} on a distributive lattice: maps between lattices that only preserve part of the lattice structure.

The simplest kind of operator is a map between lattices that respects the structure of `and' and `true', but not necessarily `or' and `false'. If this notion is understood as analogous to a linear mapping in linear algebra, then it is natural to also consider more general binary, ternary, and $n$-ary operators on lattices, which respect the structure of `and' and `true' in each coordinate, as long as the remaining coordinates are fixed. The theory of lattices with operators, and dualities for them, was developed in the second half of the 20th century, roughly in two main chunks. First, in the 1950s, by J\'onsson and Tarski, in the case of Boolean algebras, with immediate  applications to relation algebra, and the same theory was used heavily a little later and very successfully for modal logic in the form of Kripke's semantics. Second, starting from the 1980s with work by Goldblatt, and also by J\'onsson and this book's first author, duality for operators on distributive lattices came into a mature, usable form. This theory is developed in Chapter~\ref{ch:methods}, which also contains the first applications of duality theory, to free distributive lattices, quotients and subspaces, implication operators, with a focus on Heyting algebras in the last section.

In the development of the first four chapters of this book, we keep the use of category theory to a minimum. In Chapter~\ref{ch:categories}, we then set the results of the earlier chapters in the more abstract and general framework of category theory. This development then also allows us to show how Priestley's duality fits well inside a more general framework for the interaction of topology and order, which historically had shortly before been developed by Nachbin. In Chapter~\ref{chap:Omega-Pt}, we show how the various classes of topological spaces with and without order, introduced by Stone, Priestley and others, all relate to each other, and how they are in duality with distributive lattices and their infinitary variant, frames.

Chapter~\ref{chap:DomThry} and \ref{ch:AutThry} contain two more modern applications of duality theory to theoretical computer science, namely to domain theory and to automata theory, respectively.
The domain theory that we develop in Chapter~\ref{chap:DomThry} is organised around three separate results: Hoffmann-Lawson duality; the characterisation of those dcpos and domains, respectively, that fall under Stone duality; and Samson Abramsky's celebrated 1991 Domain Theory in Logical Form paper. 

The automata theory that we develop in Chapter~\ref{ch:AutThry} is organised around four separate results: finite syntactic semigroups as dual spaces and the ensuing decidability of regularity for formal languages; the free profinite monoid as the dual of the Boolean algebra of regular languages expanded with residuation operations and, more generally, topological algebras on Boolean spaces as duals of certain BAs extended by residual operations; equational characterisation results for classes of monoids using this duality; and a characterisation of those profinite monoids for which the multiplication is open.

Perhaps the most important omission of this book is the theory of canonical extensions, which was central to the already-mentioned work of J\'onsson and others, in addition to duality. While this theory is very close to both of this book's authors' hearts, and closely related to duality theory, this book is not about that. Canonical extensions do not play a big role in this book, at least not explicitly; although we will occasionally make references to canonical extensions where appropriate.
One of our hopes with this book is, however, that it will entice some readers to learn about canonical extensions. We believe the technique of canonical extensions to be complementary to, and at least as important as, duality, but so far less well-established in the literature.


\subsection*{How to use this book}
This is a textbook on spectral spaces and Stone and Priestley dualities as they have developed and are applied in various areas at the intersection of algebra, logic, and theoretical computer science. Our aim is to get in a fairly full palette of duality tools as directly and quickly as possible, then to illustrate and further elaborate these tools within the setting of three emblematic applications: semantics of propositional logics, domain theory in logical form, and the theory of profinite monoids for the study of regular languages and automata. 
The text is based on lecture notes from a 50-hour course in the \emph{Master Logique et Fondements de l'Informatique} at Paris 7, which ran in the winters of 2013 and 2014. The fact that it is based on notes from a course means that it reaches its goals while staying as brief and to the point as possible. The other consequence of its origin is that, while it is fully a mathematics course, the applications we aim at are in theoretical computer science. The text has been expanded a bit beyond what was actually said in the course, reaching research monograph level by the very end of the last two chapters~\ref{chap:DomThry} and \ref{ch:AutThry}. Nevertheless, we have focused on keeping the spirit of a lean and lively textbook throughout, including only what we need for the applications, and often deferring more advanced general theory to the application chapter where it becomes useful and relevant.

While the original course on which the book is based covered the majority of all the chapters of the book, there are several other options for its use. In particular, a basic undergraduate course on lattices and duality could treat just chapters~\ref{ch:order} through \ref{ch:priestley} and possibly selected parts of \ref{ch:methods}, \ref{ch:categories}, and/or \ref{chap:Omega-Pt}. The applications in the second part are fairly independent and can be included as wanted, although the domain theory material in Chapter~\ref{chap:DomThry} requires at least skeletal versions of Chapter~\ref{ch:categories}, and Chapter~\ref{chap:Omega-Pt} in its entirety.

The first part, Chapters~\ref{ch:order} through \ref{chap:Omega-Pt}, is a graduate level `crash course' in duality theory as it is practiced now. Chapter~\ref{ch:order} introduces orders and lattices, and in particular the distributive lattices that we will be concentrating on. Chapter~\ref{chap:TopOrd} introduces the topological side of the dualities. In this chapter, we elaborate the interaction between order and topology, which is so central to the study of spaces coming from algebraic structures. For this purpose we have bent our philosophy of minimum content somewhat by introducing the class of stably compact spaces and Nachbin's equivalent class of compact ordered spaces. We believe that this setting provides the right level of generality for understanding the connection between Stone's original duality for spectral spaces and Priestley duality. The class of stably compact spaces, being the closure of spectral spaces under continuous retracts, is also a more robust setting than spectral spaces for a number of further applications that we do not cover in this book, such as continuous domain theory and duality for sheaf representations of algebras. The basic mathematical content of Priestley duality is given in Chapter~\ref{ch:priestley}. Chapter~\ref{ch:methods} introduces the most important general methods of modern duality theory: duality for additional operations and sub-quotient duality, which then allows us to immediately give first applications to propositional logics. Chapter~\ref{ch:categories} then introduces categorical concepts such as adjunctions, dualities, filtered colimits, and cofiltered limits, which play a fundamental role in duality theory. This allows us to give a full categorical account of Priestley duality by the end of the chapter. Chapter~\ref{chap:Omega-Pt} treats the Omega-Point duality and Stone's original duality for distributive lattices and makes the relationship between these dualities and Priestley's version.

The duality theory developed in the first six chapters of the book is applied to two different parts of theoretical computer science in the last two chapters, which provide an entry into research-level material on these topics. These two chapters are independent from each other, and have indeed traditionally been somewhat separate in the literature, but our treatment here shows how both topics in fact can be understood using the same duality-theoretic techniques that we develop in the first part of the book. When using this book for a course, a lecturer can freely choose material from either or both of these chapters, according to interest. Chapter~\ref{chap:DomThry}, on domain theory, contains a duality-theoretic exposition of the solutions to domain equations, a classical result in the semantics of programming languages. Chapter~\ref{ch:AutThry} develops a duality theory for algebraic automata theory, and shows in particular how finite and profinite monoids can be viewed as instances of dual spaces of lattices with operators that we study in this book.

\subsection*{Comparison to existing literature and innovative aspects}
The first part of this book's course, Chapters~\ref{ch:order} through \ref{chap:Omega-Pt}, covers quite classical material and may be compared to existing textbooks. The closest are probably Balbes and Dwinger's \emph{Distributive Lattices} \cite{BalDwi1974}, and Davey and Priestley's \emph{Introduction to lattices and Order} \cite{DavPri2002}.  Another classical gentle introduction to the field, but focusing more on point-free topologies and frame theory than we do here, is Vickers' \emph{Topology via Logic} \cite{Vic1989}.   Of these, Balbes and Dwinger's book \cite{BalDwi1974} is probably the closest in spirit to our treatment, as it gets to the duality quickly and then applies it.
However, that book's applications to algebras of propositional logic focus on varieties that are less central today. Davey and Priestley's textbook \cite{DavPri2002} has been very successful and has in particular managed to attract a theoretical computer science readership to these topics. However, it focuses more on the lattices and order per se and the duality is covered only as one of the final crowning chapters. Davey and Priestley's book is therefore an excellent way in to ours and we recommend it as supplemental reading in case students are needing additional details or to build up mathematical maturity.

Here we aim to get the dualities in place as soon as possible and then use them. Where we differ the most from the existing books within this first part is with our emphasis on the interaction between order and topology in Chapter~\ref{chap:TopOrd}, and in placing Priestley duality within the wider context of category theory (Chapter~\ref{ch:categories}) and Omega-point duality (Chapter~\ref{chap:Omega-Pt}). We provide a textbook-level didactic account of the interaction between topology and order culminating with the equivalence between Nachbin's compact ordered spaces and stably compact spaces. In Chapter~\ref{ch:methods} we develop duality theory methods for analyzing the structure of distributive lattices and operators on them. All of these topics have become central in research in recent decades but are so far difficult to access without delving in to the specialized literature.

Many research monographs include similar material to the first part of this book, but are not explicitly targeted at readers who are first learning about the field, while this is explicitly a primary aim of our book.
Classical such monographs in the field are Johnstone's book \emph{Stone spaces} \cite{Johnstone1986} and the \emph{A compendium of continuous lattices} of Gierz et al. \cite{Getc80}, first published in 1980 and re-edited in 2003 as \emph{Continuous lattices and domains} \cite{Getc2003}, are closest in content to the first part of this book. More recently, Dickmann, Schwartz and Tressl's monograph \emph{Spectral spaces} \cite{DicSchTre2019} studies the same class of spaces as we do in this book, but coming from a more topological perspective and emphasizing less the order-theoretic aspects. Goubault-Larrecq's monograph \emph{Non-Hausdorff Topology and Domain Theory} \cite{Goubault2013}, especially in its treatment of stably compact spaces, is close in spirit to our treatment in Chapter~\ref{chap:TopOrd}, and also addresses a theoretical computer science audience, but is focused on non-Hausdorff topologies and therefore does not treat the (Hausdorff) patch topology as central, as we do here. Related to our Chapter~\ref{chap:Omega-Pt} is Picado and Pultr's monograph \emph{Frames and Locales} \cite{PicPul2012} focused on frames and point-free topology, and Chapter~\ref{chap:Omega-Pt} of this book can be used as a preparation for jumping into that work. 


The applications to domain theory and automata theory are treated  in Chapters~\ref{chap:DomThry} and \ref{ch:AutThry}, respectively. These two applications, and in particular the fact that we treat them in one place, as applications of a common theory, are perhaps the most innovative and special aspects of this book. Domain theory is the most celebrated application of duality in theoretical computer science and our treatment is entirely new. Automata theory is a new application area for duality theory and has never been presented in textbook format before. More importantly, both topics are at the forefront of active research seeking to unify semantic methods with more algorithmic topics in finite model theory. While previous treatments remained focused on the point of view of domains/profinite algebra, with duality theory staying peripheral, a shared innovative aspect of the presentations of these topics in this book is that both are presented squarely as applications of duality.

Finally, a completely original contribution of this book, which emerged during its writing, precisely thanks to our treatment of the two topics as an application of a common theory, is the fact that a notion of \emph{join-preserving at primes} turns out to be central in both the chapter on domain theory and in that on automata theory. This notion was introduced by the first-named author in 2016 in the context of automata theory and topological algebra \cite{Geh16}; its application to domain theory is new to this book. We believe this points to an exciting new direction for future research in the field that we hope a reader of this book will be inspired to take up.