\chapter{Domain Theory}\label{chap:DomThry}
In this chapter\footnote{The content of this chapter is omitted from this preprint version, but we give the reader an impression its contents.}, we develop some of the basic theory of domains and provide the key duality theoretic elements used in \cite{Abr} to solve so-called domain equations. For a more in-depth treatment we recommend \cite{Abr} and \cite{AbJu94}. The treatment here focusses on this material as an application of Stone/Priestley duality.

Section~\ref{sec:dom} introduces the notion of a \emph{domain}, which may be seen either as a special kind of poset or as a special kind of sober space, and shows that the $\Omega$-$\mathrm{pt}$ duality of Chapter~\ref{chap:Omega-Pt} restricts to the so-called \emph{Hoffmann-Lawson duality} between completely distributive lattices and domains viewed as spaces. In Section~\ref{sec:dom-Stone}, we study the intersection of this class of domains with the class of \emph{spectral spaces}, and show that, under Stone duality, the domains in this intersection correspond to a very natural class of distributive lattices. We also establish that these domains are exactly the same as the ``MUB-complete'' or ``2/3 bifinite'' domains studied in the literature \cite{Abr, AbJu94}.  In Section~\ref{sec:bifinite}, we show that the desire for a class of domains that is closed under both of these constructions naturally leads to considering \emph{bifinite} domains, which moreover are interesting objects for a duality theorist because of their self-dual nature.  Section~\ref{sec:DTLF}, the heart of this chapter, uses duality to study two constructions on domains that are central to the compositional theory of programs: powerdomains and function spaces.

\section{Domains and Hoffmann-Lawson duality}\label{sec:dom}

\section{Dcpos and domains that are spectral}\label{sec:dom-Stone}

\section{Bifinite domains}\label{sec:bifinite}

\section{Domain Theory in Logical Form}\label{sec:DTLF}


